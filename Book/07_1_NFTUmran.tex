\textbf{NFT Key Points}

\begin{enumerate}
\def\labelenumi{\arabic{enumi})}
\item
  NFTs are different from ERC-20 tokens, such as DAI or LINK, in that
  each individual token is completely unique and is not divisible.
\item
  NFTs give the ability to assign or claim ownership of any unique piece
  of digital data, trackable by using Ethereum's blockchain as a public
  ledger. An NFT is minted from digital objects as a representation of
  digital or non-digital assets
\item
  NFTs are minted through smart contracts that assign ownership and
  manage the transferability of the NFT's. When someone creates or mints
  an NFT, they execute code stored in smart contracts that conform to
  different standards, such as ERC-721. This information is added to the
  blockchain where the NFT is being managed.
\end{enumerate}

\textbf{NFT Use Cases}

\textbf{Art}

Without a doubt the recent surge of interest in NFT's during early 2021,
has largely been driven by digital art NFT's.

In 2014, New York artist
\href{https://www.mccoyspace.com/project/125/}{Kevin McCoy's
\emph{Quantum}} \emph{is widely recognised} as the first piece of art
created as an NFT. However it was only during early 2021 that art NFT's
started to gain significant attention and by the end of 2021, nearly
\href{https://www.paymentscardsandmobile.com/state-of-the-blockchain-nfts-explode-onto-scene-in-2021/}{£31b
had been spent} on NFT purchases, a considerable and exponential growth
given
\href{https://raritysniper.com/news/nfts-exploded-in-2021-with-25-billion-in-sales/}{2020
sales of \textasciitilde£71m}

High profile digital artists such as \emph{Beeple} whose
\href{https://www.forbes.com/sites/abrambrown/2021/03/11/beeple-art-sells-for-693-million-becoming-most-expensive-nft-ever/?sh=3f237d1c2448}{recent
recording break sale} of his NFT \emph{``The first 5000 days''} at
Christies (a long established British auction house, specialising in
high profile precious work of art) for £52.9m helped bring NFT's into
the public spotlight and wider give them global attention.

Art as NFT's offer the following advantages:

\begin{enumerate}
\def\labelenumi{\alph{enumi}.}
\item
  \textbf{Proof of ownership/creation:} At its most basic level, once a
  work of art is `minted' (publishing the art work as a unique token on
  the blockhain) this function as a proof of ownership and by extension
  proof of creation.
\item
  \textbf{Secure Digital Provenance} Tracking ownership and reselling of
  art; \href{https://en.wikipedia.org/wiki/Provenance}{Provenance} (or
  the chain of custody) now no longer becomes a \emph{detectives game}
  or open to fraud or misinterpretation, since provenance can increase
  the value of a piece of art (benefiting both the creator and
  collector) the use of the blockchain as an open, secure ledger is a
  far more trustworthy system than traditional methods of artistic
  provenance that were cobbled together (often consisting of documents
  spanning private \& public sale receipts, art/museum gallery
  exhibitions and private record keeping). Digital provenance provided
  when an artist mints a piece of art into an NFT allows artists and
  collectors to
\item
  \textbf{Decentralised automated royalty payments}: If a work of art is
  minted into an NFT and artist's royalty payments can be predetermined
  and automated in perpetuity via smart contracts. Once the royalty
  payment rate is set by the artist/creator, future royalties of all
  sales can be paid directly to the artist/creator account (via a
  digital wallet) without the need of a third party (traditionally a
  gallery/agent etc..). The first sale of a piece of art would often
  (but not always) benefit (financially/critically) the artist, however
  secondary and subsequent sales would historically only ever
  financially benefit the collector; the original artist would rarely
  benefit. However the use of smart contracts NFT smart contracts means
  that even if piece of art is resold 5, 10 or even a 100,000 times
  moving through 5, 10 or even a 100,000 different collectors; a
  pre-determined royalty payment rate set by the creator would still
  guarantee the artist/creator is paid directly from each and every
  future sale. If one considers that historically provenance for works
  of art can spans generations, then NFT smart contracts hold an
  incredibly potential i.e and artists descendants financially benefit
  directly from the resale of a piece of work long after the
  artist/museum's/gallery have turned to dust as long as the original
  creator's digital wallet is accessible, \emph{the blockhain becomes an
  everlasting digital patron .}
\end{enumerate}

\textbf{Computer \& Video Games}

Computer \& Video games are a huge global business. As the industry has
matured, secondary markets have emerged, most notably the `second hand'
games resale market. The rise of `retro' gaming, has demonstrated
.Despite the move to non-physical digital only' games, the demand
remains incredibly high however publisher/developers/retailers aren't
able to directly benefit from the emerging market. The potential of
\emph{video games as NFT's} presents a huge opportunity for publishers,
developers and players alike.

\begin{enumerate}
\def\labelenumi{\alph{enumi}.}
\item
  Royalty Sales on Pre-owned Games ; A predetermined proportion of any
  reasale of a used game can automated in perpetuity via smart
  contracts; once these are set by the publisher, future royalties of
  all sales can be paid directly to the publishers/developers wallets (a
  digital account) without the need of a third party (traditionally a
  retail entity). Traditionally only the initial first sale of a game
  would financially benefit the publisher/developer/retailer, secondary
  and subsequent sales would only ever financially benefit the
  purchaser, with many developers/publishers arguing this is hurting the
  wider industry through the loss of significant income generated by the
  secondary and subsequent sales, sometimes over the course of decades.
  However the use of NFT's smart contracts means that if a game is
  sold/resold through 10,000 collectors; a pre-determined royalty
  payment rate set by the publisher would still guarantee the publisher
  (and or developer/retailer) takes a proportion of any future sales.
\item
  \textbf{Monetisation of User Generated Content:} Games as a NFT's
  offer ability to monetise UGS: User generated content User generated
  states (i.e save games, user can invest hundrends/thousands of hours
  on a particular game). Games like Nintendo's Pokemon, Bungie's Destiny
  or Genshin Impact demonstrate that players invest significant amounts
  of time collecting in game digital assets; often used by players as
  mark of distinction/status symbol.
\end{enumerate}

\begin{quote}
However traditionally there has been no real utility or value that went
beyond sharing a \emph{digital badge} (i.e triumph/achievement) often
ona on social media/gamer's platform profile. NFT's offer the ideal
system for developers/publishers/players to monetise user
generated/customised content/data (such as a players unique save game
data), simultaneously developing an additional monetised ecosystem to
meet player demands (i.e certain players willing to monetise and `sell'
their invested time in a particular product/service to other players
with little time but willing to pay other players for `grinding'
(progressing laborious in game tasks)

The potential to provide publishers/developers with an additional
long-term income stream, providing a better ROI on computer \& video
game development, which in many instances can cost hundrends of millions
in development costs spanning 5/10 years, is undeniable.
\end{quote}

\begin{enumerate}
\def\labelenumi{\alph{enumi}.}
\setcounter{enumi}{2}
\item
  Play to earn revenue models.
\item
  Monetizing In game collectibles: customisable in game assets (vanity
  items such as cosmetic character skins/clothing or collectible items
  that offer player advantages(new weapons/vehicles/mods etc,..)
\end{enumerate}
