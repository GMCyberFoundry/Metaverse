Distributed ledger technology (DLT) is a data structure distributed across multiple managing stakeholders. A subset of DLT is blockchain, which is a less efficient, immutable data structure with a slightly different trust model. Rauchs et al. of the Cambridge Centre for Alternative Finance provide a detailed taxonomy and conceptual framework \cite{rauchs2018distributed}. It can be seen in their paper that the definitions are somewhat unclear in literature.\par
DLT, and especially blockchain, are rapidly gaining ground in the public imagination, within financial technology companies (FinTech), and in the broader corporate world. \par
The technology and the global legislative response are somewhat immature, and misapplications of both technologies are commonplace. \par
Distributed trust models emerged from cryptography research in the 1970s when Merkle, Diffie, and Hellman at Stanford worked out how to \href{https://medium.com/swlh/understanding-ec-diffie-hellman-9c07be338d4a}{send messages online} without a trusted third party \cite{diffie1976new,merkle1978secure}.\par
Soon after the 1980s saw the emergence of the cypherpunk activist movement, as a reaction to the emerging surveillance state \cite{burnham1983rise, chaum1985security}. These early computer scientists in the USA saw the intersectionality between information, computation, economics, and personal freedom \cite{lavoie1990prefatory}. Online discussion in the early nineties foresaw the emergence of trans-national digital markets, what would become the WWW \cite{salinCosts, cypherPunkMailList}. The issues of privacy %(https://nakamotoinstitute.org/static/docs/cypherpunk-manifesto.txt) 
 and the exchange of digital value (digital / ecash) %(https://www.wired.com/1994/12/emoney/  https://www.cs.ru.nl/~jhh/pub/secsem/chaum1985bigbrother.pdf) 
 were of foremost importance within these discussions %(https://www.wired.com/1994/12/emoney/), 
 and while privacy was within reach thanks to \href{https://www.openpgp.org/about/history/}{``public/private key pairs''}, 
 ecash proved to be a more difficult problem. \par
Adam Back's 1997 `hashcash' \cite{back2002hashcash} paved the way for later work by introducing the concept of `proof of work'. This was built upon by Dai \cite{dai1998b}, Szabo \cite{szabo1997formalizing}, Finney \cite{callas1998openpgp}, and Nakamoto amongst others. In all it took 16 years of collaboration on the mailing lists (and dozens of failed attempts) to attack the problem of trust-minimised, distributed, digital cash. The culmination of these attempts was Bitcoin \cite{Nakamoto2008}. This is illustrated by Dan Held in Figure \ref{fig:prehistory}. This is now a wider ecosystem of technologies and societal challenges \ref{fig:bitcointopics}. 

\begin{figure}
  \centering
    \includegraphics[width=\linewidth]{prehistory}
  \caption{Dan Held: \href{https://www.danheld.com/blog/2019/1/6/planting-bitcoinsoil-34}{Bitcoin prehistory} used with permission.}
  \label{fig:prehistory}
\end{figure}

\begin{figure}
  \centering
    \includegraphics[width=\linewidth]{bitcointopics}
  \caption{\href{https://twitter.com/djvalerieblove/status/1514703620272394243/photo/1}{Bitcoin Topics} used with permission @djvalerieblove.}
  \label{fig:bitcointopics}
\end{figure}

There is enormous complexity and scope, as seen in Figure \ref{fig:venn}, and yet genuinely useful products are elusive.
\begin{figure*}[ht]\centering % Using \begin{figure*} makes the figure take up the entire width of the page
	\includegraphics[width=\linewidth]{venn}
	\caption{\href{https://unchained.com/blog/blockchain-spectrum/}{Intersecting disciplines}. Reused with permission \href{https://unchained.com/}{Dhruv Bansal}}
	\label{fig:venn}
\end{figure*}
It is surprisingly hard to pin down a simple explanation for the features which define a blockchain. These ``key takeaway'' \href{https://www.investopedia.com/terms/b/blockchain.asp}{from Investopedia} are a neat summary however.\par
\textit{\begin{itemize} \item Blockchain is a specific type of database. \item It differs from a typical database in the way it stores information; blockchains store data in blocks that are then chained together. \item As new data comes in it is entered into a fresh block. Once the block is \href{https://bits.monospace.live/}{filled with data} it is chained onto the previous block, which makes the data chained together in chronological order. \item Different types of information can be stored on a blockchain but the most common use so far has been as a ledger for transactions. \item In Bitcoin’s case, blockchain is used in a decentralized way so that no single person or group has control—rather, all users collectively retain control. \item Decentralized blockchains are ``append only''. In effect this means that the data entered becomes irreversible over time. For Bitcoin, this means that simple economic transactions are permanently recorded and viewable to anyone. \end{itemize}}
In principle blockchains provide a \textbf{differentiated trust (and risk) model}. With a properly distributed system a blockchain can be considered ``trust-minimised'', though certainly not risk minimised. This is important for some, but not all people. There is not much emboldening of text within this book. If you start to question the whole reason for this `global technology revolution' then it always comes back to those three words. \par
It can in fact be argued that the whole concept of distributed cryptographic blockchains is somewhat strained, as the vast majority of the technology offerings are not properly distributed, and ``there are many scenarios where traditional databases should be used instead''\cite{casino2019systematic}.\par
\section{What's this for sorry?}
The proponents of blockchains argue, that in an era when data breaches and corporate financial insolvency intersect with a collapse in trust of institutions, it is perhaps useful to have an alternative model for storage of data, and value. That seems like a lot of effort for a questionable gain. It's far more likely it's simply speculation.\par 
While writing this book the questions of `what is this \textit{really for} and how can it possibly be worth it', came up again and again. In truth it's a very difficult question, without a clear enough answer. It's beyond the scope of this book to figure this out properly, but references to advantages and disadvantages will be made throughout.\par  
It seems that the engineers who created Bitcoin wanted very much to solve a technical problem they saw with money (from their understanding of it), and the transmission of money digitally. As the scale and scope have increased so has the narrative evolved, but it's never really kept pace with the level of the questions posed. \par
A cost benefit analysis that excludes speculative gains seems to fail for pretty much all of blockchain/DLT. Bitcoin is more subtle as it possibly \textit{can} circumvent the legacy financial systems. This still leaves huge questions. To quote others in the space, is Bitcoin now the iceberg or the life raft? \par 
For the most cogent defence of the technology as it stand for this moment, Gladstein offers a vision for the asset class, in the 87\% of the world he says don't have access to the benefits enjoyed by the developed west \cite{gladsteincheck2022}. To further contextualise this \href{https://www.youtube.com/watch?v=BRQIMjZLMDk}{Mike Novogratz} claims the following adoption figures. It is conceivable this happens irrespective of `usefulness'.\par
\begin{table}[]
\begin{tabular}{|l|r|l|r|}
\hline
\textbf{Country} & \textbf{\begin{tabular}[c]{@{}r@{}}\% of pop own\\ crypto\end{tabular}} & \textbf{Country} & \textbf{\begin{tabular}[c]{@{}r@{}}\% of pop own\\ crypto\end{tabular}} \\ \hline
Ukraine          & 12.7                                                                    & South Africa     & 7.1                                                                     \\ \hline
Russia           & 11.9                                                                    & Nigeria          & 6.3                                                                     \\ \hline
Venezuela        & 10.3                                                                    & Colombia         & 6.1                                                                     \\ \hline
Singapore        & 9.4                                                                     & Vietnam          & 6.1                                                                     \\ \hline
Kenya            & 8.5                                                                     & Thailand         & 5.2                                                                     \\ \hline
USA              & 8.3                                                                     & United Kingdom   & 5.0                                                                     \\ \hline
India            & 7.3                                                                     & Brazil           & 4.9                                                                     \\ \hline
\end{tabular}
\end{table}
Gladstein's is a carefully developed and well researched book, but is written from the perspective of (just) Bitcoin `being the raft'. Later in this book we will consider if it might be the iceberg, but this is not the domain expertise we offer in this book.\par
Thanks to a natural fit with strong encryption, and innate resistance to censorship by external parties, these systems do lend themselves well to `borderless' applications, and are somewhat resistant to global regulation (for good or ill). This provides us an excellent use case to explore for metaverse applications, and this will be the focus.
\section{A panoply of tech}
Within DLT/blockchain there seem to be as many opinions on the value of the technology as there are implementations. A host of well engineered open source code repositories makes the cost of adoption relatively low, while order of magnitude more that look very similar make the risks incredibly high.\par
There are thousands of different `chains' and many more tokens which represent value on them. A majority of these are code forks of earlier projects. Most \href{https://99bitcoins.com/deadcoins/}{are defunct} yet still have some residual `value' locked up in them as a function of their `distributed' tokens. \par 
Because the space is comparatively new, subject to \href{https://www.esma.europa.eu/press-news/consultations/call-evidence-dlt-pilot-regime}{scant regulation}, and often open source, it is possible to clone a github, change a few lines of code, and front it with a website in order to create `scams', and this happens frequently \cite{golumbia2020cryptocurrency}.\par
The following sections give an overview of the major strands of the technology. First is Ethereum, mainly to discount it's use for our needs, and move on to more appealing options.
\section{Ethereum}
Ethereum \cite{buterin2013ethereum} is the second most \href{https://www.crypto51.app/}{secure} public blockchain (\href{https://howmanyconfs.com/}{by about 50\%})\cite{sayeed2019assessing}, and second most valuable by \href{https://coinmarketcap.com/}{market capitalisation} (though this comparison is somewhat stretched). It is the natural connection from Web3 to the rest of the book, so it will be considered first.\par
It is touted as `programmable money'. It, unlike bitcoin, is (\href{https://hackernoon.com/turing-completeness-and-the-ethereum-blockchain-c5a93b865c1a}{nearly}) Turing complete \cite{petzold2008annotated}, able to run a \href{https://ethereum.org/en/developers/docs/evm/}{virtual machine} within the distributed network (albeit slowly), and can therefore process complex transactional contracts in the settlement of value. This has given rise to the new field of `distributed finance', or DeFi (described later), alongside many interesting trust-minimised immutable ledger public database ideas. \par
There are trade-offs and problems with Ethereum (Eth/Ether) which currently increase the `participation floor' and make the network far less suitable for entry level business-to-business use. The ledger itself being a computational engine, with write only properties, is enormous. Specialist cloud hardware is required to run a full node (copy of the ledger), and partial nodes are the norm. Even partial nodes are run chiefly by one specialist cloud provider (\href{https://consensys.net/blog/news/why-infura-is-the-secret-weapon-of-ethereum-infrastructure/}{Infura}), which has recently been forced to \href{https://finance.yahoo.com/news/metamask-infura-block-certain-areas-173749914.html}{exclude Venezuela} from the network. Critics of the project point to this vulnerability to outside influence as an existential threat to the aims of the technology. If it can be censured, then what advantage is there over the \href{https://protos.com/consensys-lawsuit-jpmorgan-owns-critical-ethereum-infrastructure/}{founders} simply running a high speed database to the same purpose? \par
This is a function of the so called `scalability trilema' \cite{hafid2020scaling}, in which it seems that only two features from the list of decentralization, scalability or security can be chosen for blockchains \cite{bonneau2015sok}.\par
Moreover the network is centrally controlled by its creator and the `miners'. There is a strong case to answer that Eth is \href{https://blog.mollywhite.net/blockchains-are-not-what-they-say/}{neither distributed}, nor trustless, and in fact therefore fails to be differentiated from a DLT, undermining some of it's claims. The history of Ethereum is a fascinating case study in human greed. By the time the whitepaper had it's first limited release, Bitcoin (covered next) had already passed \$1000 per token. This led to the creators ambitions for a `fair release' of tokens being voted down by powerful funders, leading to the explosion of similarly structured `pre-mined' coins in the ICO craze, which followed on the Ethereum network. Laura Shin is possibly the most experienced journalist and author in the space and has covered this crazy era in her book `The Cryptopians' \cite{cryptopians}. It's a tough read for the newcomer though, perhaps finish this primer first!\par
With that said there are many talented developers doing interesting work on the platform, and innovation is fast paced. It is entirely normal for technology projects to launch their distributed ledger idea on and within the Ethereum network. These generate tradable `ERC-20' tokens, which can accrue value or demonstrate smart contract utility  (based on the \href{https://soliditylang.org/}{Solidity} programming language). Because the value locked and generated in the Ethereum platform comes not just from the ETH token, but all the ERC technologies built upon it, there are hundreds of billions of pounds `within' the network. Most of this money is pure market speculation (as is the case across blockchains). Many analysts cannot see this as anything but a speculative bubble, with all the predictable crash yet to come. This can be seen in the context of other bubbles in Figure \ref{fig:etherbubble}. It seems that most of the projects in crypto more generally, but certainly with ETH and the NFTs within it are a new kind of social gambling, where online communities can reinforce groupthink around their speculative choices. With all this said most of the couple of million people who use NFTs use Ethereum, and this market of creators and consumers is to be brought into a mixed reality space then they will need a way to bring their objects with them. 
\begin{figure}
  \centering
    \includegraphics[width=\linewidth]{etherbubble}
  \caption{Ethereum is thought to look like a speculative bubble. Rights requested}
    \label{fig:etherbubble}
\end{figure}

Such is the level of nefarious activity on these networks (within Ethereum) that they have a poor reputation, and are difficult to audit, launch, and maintain. The overriding problem of using a blockchain for utility applications (rather than just as money) is that people can, and will, simply lie for criminal purpose when entering data into the ledger. It is far more likely that Ethereum is simply a speculative bubble than any of the claims for utility being born out. Add to that \href{https://advisor.morganstanley.com/daron.edwards/documents/field/d/da/daron-edwards/Cryptocurrency_201__What_is_Ethereum_.pdf}{Morgan Stanleys recent assertion} that Ethereum is itself threatened by newer contender chains and it's future becomes unclear. The report correctly identifies that ``High transaction fees create scalability problems and threaten user demand. High costs make Ethereum too expensive for small-value transactions.''. It is this high cost of use that most excludes the ERC-20 networks from our consideration.
\subsection{Mining and Gas}
Ethereum has a significant barrier to entry because of high fees to use the network. The system is Turing complete; able to programatically replicate any other computational system. This includes endless loops in code, so it is trivial to lock up the computational bandwidth of the whole system, in a smart contract commitment, through a web wallet. \par 
To mitigate this existential `denial of service attack' the `gas' system demands that users spend some of their locked up value to operate on the network. In this way a transaction loop would quickly erode the available gas and stop looping. As the popularity of the system has grown, so too have the gas fees. It can sometimes cost hundreds of dollars to do a single transaction, though it is typically a few tens of dollars. Appallingly if the user pitches their mining fee offer too low, then the money gets spent anyway! \href{https://fees.wtf/#/}{A website} just plucks random Ethereum addresses out of the aether to show you the level of this expense for participants. People can even \href{https://opensea.io/collection/fees-wtf-nft?search[sortAscending]=false&search[sortBy]=PRICE}{buy NFTs} of the worst examples of these wastes, wasting more money, because of course they can! This is a huge problem for potential uses of the network. \par
It is currently a proof of work system like Bitcoin (this is described in the next section), and has a \href{https://news.trust.org/packages/cryptocurrency-and-climate/}{commensurate energy footprint} to secure the network. It also ties up global supply of PC graphics cards used for it's mining model, making them far more expensive. This has generated ill will in the global gamer community for instance, and damped the ambitions of NFT developers because of their inability to sway this crucial customer demographic. Much more on this later.\par
\subsection{Upgrade roadmap}
Part of the challenge Ethereum faces is wrapped up with it's complex token emission schedule. This is the rate at which tokens are generated and `burnt' or destroyed in the network. The total supply of tokens is uncertain, and both emission and burn schedules are regularly tinkered with by the project. The changes to the rate at which ETH are generated can be seen in Figure \ref{fig:ethemission}.
\begin{figure}
  \centering
    \includegraphics[width=\linewidth]{ethemission}
  \caption{The rate of token generation has changed unpredictably over time. Rights requested}
  \label{fig:ethemission}
\end{figure}
In addition, a recent upgrade (EIP-1559) results in tokens now being burnt at a higher rate than they are produced, deliberately leading to a diminishing supply. In theory this increases the value of each ETH on the network at around 3\% per year. It's very complex, with impacts on transaction Fees, waiting time, and consensus security, as examined by Liu at al. \cite{liu2022empirical}. Addionally, there is now talk (by \href{https://time.com/6158182/vitalik-buterin-ethereum-profile/}{Butlerin}, the creator of Ethereum) of extending this burn mechanism \href{https://ethresear.ch/t/multidimensional-eip-1559/11651}{further into the network}.\par
Ethereum was designed from the beginning to move to a `proof of stake' model where token holders underpin network consensus through complex automated voting systems based upon their token holding. This is now called \href{https://blog.ethereum.org/2022/01/24/the-great-eth2-renaming/}{Ethereum Consensus Layer}. Proof of stake has problems in that the majority owners `decide' the truth of the chain to a degree, and must by design have the ability to over-ride prior consesnsus choices. This has serious implications for malicious actors who have sufficient control of the existing history of the chain. Like much of the rest of `crypto' the proposed changes will concentrate decisions and economic rewards in the hands of major players, early investors, and incumbents. This is a far cry from the stated aims of the technology. The move to proof of stake has recently earned it the \href{https://www.technologyreview.com/2022/02/23/1044960/proof-of-stake-cryptocurrency/}{MIT breakthrough technology award}, despite not being complete. It's clearly a technology which is designed to innovate at the expense of predictability. This might work out very well for the platform, but right now the barrier to participation (in gas fees) is so high that we do not intend for Ethereum to be in scope as a method for value transfer within metaverses.\par


\section{Bitcoin}
The first blockchain was the Bitcoin network \cite{Nakamoto2008}, some two decades after Haber et al. first described the idea \cite{haber1990time}. Prior to Bitcoin these structures were called `timechains' \cite{nakamoto2018}. It can be considered a triple entry book keeping system \cite{ijiri1986framework, faccia2019accounting}, the first of it's kind, integrating a `provable' timestamp with a transaction ledger, solving the ``double spend problem'' \cite{chohan2021double, perez2019double}. Some see this as the first major innovation in ledger technology since double entry was codified in Venice in fourteen seventy five\cite{sangster2015earliest}. \par
It was created pseudonomously by an individual or group calling themselves `Satoshi Nakamoto' in 2009, as a direct response to the perceived mishandling of the 2008 global financial crisis \cite{nakamoto2018}, with the stated aim of challenging the status quo, with an \href{https://world.hey.com/dhh/i-was-wrong-we-need-crypto-587ccb03}{uncensorable} technology, to create a money which could not be \href{http://p2pfoundation.ning.com/forum/topics/bitcoin-open-source}{debased by inflation policy}. \par
The \href{https://en.bitcoin.it/wiki/Genesis_block}{``genesis block''} which was hard coded at the beginning of the `chain' contains text from The Times newpaper detailing the second bank bailout.\par 
There will only ever be (\href{https://blog.amberdata.io/why-the-bitcoin-supply-will-never-reach-21-million}{just short of}) 21 million bitcoins issued, of which around 19 million have already been minted, and around 4 million lost forever. This `hard money' absolute scarcity is a strong component of the Bitcoin meme landscape. These are basically arbitrary figures though; a combination of the issuance schedule, and an \href{https://plan99.net/~mike/satoshi-emails/thread1.html}{`educated guess'} by Nakamoto: \cite{nakamoto2018}\par 
\textit{''My choice for the number of coins and distribution schedule was an educated guess.  It was a difficult choice, because once the network is going it's locked in and we're stuck with it.  I wanted to pick something that would make prices similar to existing currencies, but without knowing the future, that's very hard.  I ended up picking something in the middle.  If Bitcoin remains a small niche, it'll be worth less per unit than existing currencies.  If you imagine it being used for some fraction of world commerce, then there's only going to be 21 million coins for the whole world, so it would be worth much more per unit.''}\par
In theory there is no \href{https://www.forbes.com/sites/peterizzo/2021/09/29/against-cryptocurrency-the-ethical-argument-for-bitcoin-maximalism/?}{barrier to access}, and \href{https://www.coindesk.com/layer2/2022/02/16/why-bitcoin-is-a-tool-for-social-justice/}{equality of opportunity} to accumulate and save over long periods. This is not true of chains and tokens since, which lock up some of their value for seed investors to cash out later. None of the blockchains since are decentralised in the same way \cite{selvam2021blockchain}. Bitcoin was probably a \href{https://danhedl.medium.com/bitcoins-distribution-was-fair-e2ef7bbbc892}{singular event}.\par
Each Bitcoin can be divided into 100 million satoshis (sats), so anyone buying into Bitcoin can buy a thousandth of a pound, assuming they can find someone willing to transact that with them. \par
Satoshi Nakamoto (the name of the publishing entity) \href{https://bitcoinmagazine.com/technical/what-happened-when-bitcoin-creator-satoshi-nakamoto-disappeared}{disappeared from the forums} forever in 2010. Bitcoin has the marks of cypherpunks and anarcho capitalism. The IMF has recently conceded that the Bitcoin \href{https://blogs.imf.org/2022/01/11/crypto-prices-move-more-in-sync-with-stocks-posing-new-risks/}{poses a risk} to the traditional financial systems, so it could be argued that it is succeeding in this original aim.\par
Although there were some earlier experiments (hashcash, b-money etc), Bitcoin is the first viably decentralised `cryptocurrency'; the network is used to \href{https://www.aier.org/article/why-does-bitcoin-have-value/}{store economic value} because it is judged to be secure and trusted. It is a singular event in that it became established at scale, such that it could be seen to be a fully distributed system, without a controlling entity. This is the differentiated trust model previously mentioned. This relative security is the specific unique selling point of the network. It is many times more secure than all the networks which came after based on a like for like comparison of \href{https://howmanyconfs.com/}{transaction `confirmations'}. This network effect of Bitcoin is a compounding feature, attracting value through the security of the system. It is deliberately more conservative and feature poor, preferring instead to \href{https://bips.xyz/}{add to it's feature set} slowly, preserving the integrity of the value invested in it over the last decade. At time of writing it is a \href{https://fiatmarketcap.com/}{top quartile} largest global currency and has settled over \$12 trillion Dollars in 2021, though Makarov et al. contest this, citing network overheads, and speculation \cite{makarov2021blockchain}. Institution grade `exchange tradable funds' which allow investment in Bitcoin are available throughout the world, and the native asset can be bought by the public easily through apps in all but a handful of countries as seen in Figure \ref{fig:settled2021}. \par
\begin{figure}
  \centering
    \includegraphics[width=\linewidth]{settlement2021}
  \caption{\href{https://twitter.com/glxyresearch/status/1469039427028664320?}{Growth in settlement} value on the Bitcoin network.}
  \label{fig:settled2021}
\end{figure}
Only around 7 transactions per second can be settled on Bitcoin. The native protocol does not scale well, and this is an inherent trade-off as described by Croman et al. in their positioning paper on public blockchains \cite{croman2016scaling}. Over time, competition for the limited transaction bandwidth drives up the price to use the network. This effectively prices out small transactions, even locking up some value below what is a termed the '\href{https://github.com/bitcoin/bitcoin/blob/v0.10.0rc3/src/primitives/transaction.h#L137}{dust limit}' of unspent transactions too small to ever move again \cite{delgado2018analysis}. \par
Bitcoin has developed quickly, with a \href{https://phemex.com/blogs/crypto-bitcoin-s-curve-adoption-curve}{faster adoption} than even the internet itself. It is now a mature ecosystem, and is seeing adoption as a \href{https://bitcointreasuries.net/}{corporate treasury asset}. \par
Adoption by civil authorities is increasing, and legislators the world over are being forced to \href{https://www.politico.com/news/2022/01/16/bitcoin-crashes-the-midterms-527126}{adopt a position}. Many city treasuries have \href{https://www.bloomberg.com/news/articles/2022-01-14/rio-de-janeiro-wants-to-become-brazil-s-cryptocurrency-capital}{added it} to their balance sheet. The Swiss city of Lugano is launching a \href{https://twitter.com/Stadicus3000/status/1499656424422526977}{huge initiative} alongside Tether. It is already legal tender in the country of El Salvador\cite{oxford2021salvador}, and will be soon in \href{https://www.forbes.com/sites/ninabambysheva/2022/04/07/two-new-territories-are-adopting-bitcoin/?sh=7f014ed2499a}{Madeira and Roatán island}. This will be explored more later. Global asset manager ``Fidelity'' wrote the following in their \href{https://www.fidelitydigitalassets.com/articles/2021-trends-impact}{2021 trends report}.\par
\textit{``We also think there is very high stakes game theory at play here, whereby if Bitcoin adoption increases, the countries that secure some bitcoin today will be better off competitively than their peers. Therefore, even if other countries do not believe in the investment thesis or adoption of bitcoin, they will be forced to acquire some as a form of insurance. In other words, a small cost can be paid today as a hedge compared to a potentially much larger cost years in the future. We therefore wouldn't be surprised to see other sovereign nation states acquire bitcoin in 2022 and perhaps even see a central bank make an acquisition.''}\par
\subsection{The Bitcoin Network Software}
There isn't a single GitHub which can be considered the final arbiter of the development direction, because it is a distributed community effort with some \href{https://decrypt.co/66740/who-are-the-fastest-growing-developer-communities-in-crypto}{400 developers} out of a wider `crypto' pool of around 9000 contributors. \href{https://bitcoinops.org/en/newsletters/2021/12/22/}{Development and innovation continues} but there is an emphasis on careful iteration to avoid damage to the network. Visualisation of code commitments to the various open source software repositories can be seen at \href{https://www.youtube.com/channel/UC4DT4qudqogkmbqVAQy8eFg/videos}{Bitpaint youtube channel} and in Figure \ref{fig:gource}.\par
\begin{figure*}[ht]\centering % Using \begin{figure*} makes the figure take up the entire width of the page
	\includegraphics[width=\linewidth]{gource}
	\caption{\href{https://github.com/bitpaint/bitcoin-gources}{Bitpaint}: Contributions to the Bitcoin ecosystem. Reused with permission.}
	\label{fig:gource}
\end{figure*}
\href{https://github.com/bitcoin/}{Bitcoin core} is the main historical effort, but there are alternatives (\href{https://github.com/libbitcoin/libbitcoin-node/wiki}{LibBitcoin in C++}, \href{https://github.com/btcsuite/btcd}{BTCD in Go}, and \href{https://bitcoinj.github.io/getting-started}{BitcoinJ in Java}), and as innovation on layer one slows, attention is shifting to codebases which interact with the base layer asset. Much more on these later.
\subsection{Mining and Energy concerns}
Bitcoin mining is the process of adding public transactions into the ledger, in return for two economic rewards, paid in Bitcoin. These are the mining fee, and the block reward. The transactions which are added into the next `block' of the chain are selected preferentially based on the fee they offer, which is up to the user trying to get their transaction into the chain. This can be within the next 10 minutes (next block), or a gamble out toward 'never' depending how competitive the network is at any time. Miners try to find a sufficiently low ``magicnumber'' resulting from a cryptographic hash function \cite{rogaway2004cryptographic}, and upon finding it, they can take their pre-prepared `block' of transactions sourced from their local queue (mempool), and add it into the chain, for confirmation by other miners. In return they take all the fees within that mined block, and whatever the block reward is at the time. When the network started the block reward was 50 Bitcoin, but has \href{https://ma.ttias.be/dissecting-code-bitcoin-halving/}{halved} repeatedly every 210,000 blocks (four years) and now stands at 6.25 BTC. The rate of mining is kept roughly at one block every 10 minutes, by a difficulty adjustment every 2016 blocks (2 weeks). This in a complex interdependent mechanism and is explained very well in \href{https://bitcoinmagazine.com/technical/how-mining-protects-the-bitcoin-network}{this article}. These components are explained in slightly more detail later.\par
Bitcoin uses a staggering amount of energy to secure the blockchain, and this \href{https://www.edmundconway.com/bitcoin-money-and-the-planet/}{has climate repercussions}. It is an \href{https://www.ruetir.com/2022/03/18/riot-whinstone-the-bitcoin-farm-with-100000-computers-that-uses-excess-energy-from-an-oil-platform-to-mine-cryptocurrencies-ruetir/}{industrial scale} global business with `mining companies' investing \href{https://ir.marathondh.com/news-events/press-releases/detail/1272/marathon-digital-holdings-bitcoin-mining-fleet-to-reach}{hundreds of millions of pounds} at a time in specialist \href{https://en.wikipedia.org/wiki/Application-specific_integrated_circuit}{ASIC} mining hardware and facilities. The latest purpose designed Intel chip \href{https://www.intel.com/content/www/us/en/newsroom/opinion/thoughts-blockchain-custom-compute-group.html#gs.pd9ofu}{touts} both Web3 and metaverse applications. This is Adam Back's ``proof of work'',  and is essential to the technology, and is still thought to be the \href{https://www.truthcoin.info/blog/pow-cheapest/}{best available option}. \href{https://ccaf.io/cbeci/index}{The Cambridge Bitcoin Energy Consumption Index} monitors this energy usage.\par
Such businesses can mine a Bitcoin for around \$5k-\$10k per coin, so the profit margins \href{https://www.nicehash.com/profitability-calculator}{are considerable} (based on 30-40 Joule/terahash and power rate less than 5 cents/kilowatt hour and excluding hardware costs). This is not to say that all mining is, or should be, so concentrated. Anyone running the hashing algorithm can \href{https://twitter.com/ckpooldev/status/1485585814419812356}{get lucky} and claim the block reward. PoW ties the value of the `money' component of Bitcoin directly to energy production. This is not a new idea. Henry Ford proposed an intimate tie between energy and money to create a separation of powers from government, as can be seen in Figure \ref{fig:energyNYT}.\
\begin{figure}
  \centering
    \includegraphics[width=\linewidth]{energyNYT}
  \caption{\href{https://www.nytimes.com/1921/12/06/archives/mr-fords-energy-dollar.html}{Intimate tie between energy and money, Henry Ford}}
  \label{fig:energyNYT}
\end{figure}
The potential ecological footprint of the network has always been a concern; Hal Finney himself was \href{https://twitter.com/halfin/status/1153096538}{thinking about this issue} with a mature Bitcoin network as early as 2009, and \href{https://satoshi.nakamotoinstitute.org/posts/bitcointalk/threads/167/#35}{a debate} on the Bitcoin mailing lists called the mining process ``thermodynamically perverse'' \par
\href{https://electricmoney.org/}{Proponents of the technology} say that the balance shifted dramatically in 2021 with China outright banning the technology; this has forced the bulk of the energy use away from `dirty coal' as seen in Figure \ref{fig:miningshare}. Some analysts \href{https://docs.google.com/document/d/1N2N-5jY00cmteoY_puWI9oosM1foa4EQqsO1FFfIFR4/edit}{have proposes mitigations} \cite{cross2021greening}. As a worked example of \href{https://docs.google.com/spreadsheets/d/15e_a-D3x4fv3tglEzFmQ6TLQx0fZe6-iKO9Fc9SyISQ/edit#gid=0}{Cross and Bailey's proposal} a retail investor owning 1 BTC would have to buy around 700 shares of `CleanSpark' mining company (CLSK) to make their \href{https://docs.google.com/spreadsheets/d/1r32T8p_PHTP8S781u7PhPSwehLx2VcJTaJJKesMswD0/edit#gid=0}{holding completely neutral}.  Some more strident voices suggest that \href{https://medium.com/@magusperivallon/a-financial-hail-mary-for-the-climate-an-argument-for-bitcoin-adoption-9c58e707d0}{`ending financialisation' through use of Bitcoin} may be net positive for the environment at a macro level. Indeed it may \href{https://www.newsweek.com/bitcoin-mining-americas-most-misunderstood-industry-opinion-1669892}{provide a route} to support \href{https://mobile.twitter.com/DSBatten/status/1514072998881665027}{electrifying everything} through deployment of \href{https://lancium.com/solutions/}{flexible base load}, and local subsidy initiatives (as may be \href{https://braiins.com/blog/bitcoin-mining-the-grid-generators}{happening in Texas})\cite{griffith2021electrify}.\par Recently for example Baur and Oll found that \textit{``Bitcoin investments can be less carbon intensive than standard equity investments and thus reduce the total carbon footprint of a portfolio.''}\cite{baur2021bitcoin}. Perhaps of note for the near future is that KPMG whose investment was mentioned in the introduction also matched their position in the space with equivalent  carbon offsets. This may provide an investment and growth model for others.
\begin{figure}
  \centering
    \includegraphics[width=\linewidth]{miningshare}
  \caption{Hash rate \href{https://ccaf.io/cbeci/ining_map}{suddenly migrates} from China [Reuse rights requested]}
  \label{fig:miningshare}
\end{figure}
The power commitment to the network is variously projected \href{https://www.nature.com/articles/s41558-018-0321-8}{to increase}, or \href{https://assets.website-files.com/614e11526f6630959fc98679/616df63a27a7ec339f5e6a80_NYDIG-BitcoinNetZero_SML.pdf}{level off over time}, but certainly not decrease. The \href{https://www.forbes.com/sites/martinrivers/2022/04/03/is-bitcoin-really-that-bad-for-the-environment/?sh=6a3203427143}{industry now argues} that economic pressures mean that most of the `hashrate' is \href{https://bitcoinminingcouncil.com/q4-bitcoin-mining-council-survey-confirms-sustainable-power-mix-and-technological-efficiency/}{generated by renewable energy}\cite{blandin20203rd}. As a recent example of this trend Telsa (Elon Musk), Block (Twitters Jack Dorsey), and Blockstream (Adam Back) are teaming up to \href{https://www.cnbc.com/2022/04/08/tesla-block-blockstream-to-mine-bitcoin-off-solar-power-in-texas.html}{mine with solar energy} in Texas.\par 
There is growing interest and adoption of so called ``stranded energy mining''  which cannot be effectively transmitted to consumers, and is thereby sold at a huge discount while also \href{https://www.renewableenergyworld.com/wind-power/900mw-wind-farm-to-power-bitcoin-mining-operation/}{developing power capacity} \cite{bastian2021hedging}, and/or \href{https://www.bloomberg.com/news/articles/2022-03-24/exxon-considers-taking-gas-to-bitcoin-pilot-to-four-countries}{reducing the carbon} of existing infrastructure. A \href{https://twitter.com/thelaserlife/status/1511354396705452035}{pseudonymous twitter user} puts this well:\par 
\textit{``Eventually it'll seem obvious – having a mechanism to instantly monetize energy at its source was the missing ingredient to massively scale up green energy production.''}. The actual markets economics of this are far from simple, but are well explained by \href{https://www.whatbitcoindid.com/podcast/bitcoin-energy-markets}{Connell in a podcast}.\par
The most cited example of building capacity before grid connection is El Salvador's `volcano mining' proposal, which is supporting their national power infrastructure plans. A more poignant example is the \href{https://www.timesunion.com/news/article/Mechanicville-hydro-plant-gets-new-life-16299115.php}{Mechanicville hydro plant in the USA}. The refurbishment of this 123 year old power plant is being funded by Bitcoin mining. This is the \href{https://www.lynalden.com/bitcoin-energy/}{``buyer of last resort''} model first \href{https://squareup.com/us/en/press/bcei-white-paper}{advanced by Square Inc}. Critics highlight the potential \href{https://www.fitchratings.com/research/us-public-finance/crypto-mining-poses-challenges-to-public-power-utilities-24-01-2022}{impact of mining on local energy} prices\cite{benetton2021cryptomining}.\par
\href{https://www.youtube.com/watch?v=6LP8G-oZnEs}{The debate} whether this consumption is `worth' it is \href{https://www.utilitydive.com/news/bitcoin-mining-as-a-grid-resource-its-complicated/617896/}{complex} and \href{https://www.aei.org/technology-and-innovation/no-hearing-on-bitcoins-energy-use-is-complete-without-nic-carter/}{rapidly evolving}. A useful example of this is the \href{https://www.zerohedge.com/crypto/questionable-ethics-anti-bitcoin-esg-junk-science}{online pushback} to an academic article by de Vries et al \cite{de2022revisiting}, and this well considered \href{https://twitter.com/jyn_urso/status/1508899761319038983}{Twitter thread} by climate scientist Margot Paez.\par 
This stuff is existentially important to the whole technology. Is a trillion dollar asset which \href{https://www.theheldreport.com/p/bitcoin-vs-gold}{potentially replaces} the money utility of gold, but doesn't need to be stored under guard in vaults (Figure \ref{fig:goldmanVgold}), worth the equivalent power consumption of clothes dryers in North America? Probably not with the current level of adoption, but this is an experiment in replacing global money.\par
\begin{figure}
  \centering
    \includegraphics[width=\linewidth]{goldmanvgold}
  \caption{Goldman suggest growth opportunity and potential demonetisation of gold?}
  \label{fig:goldmanVgold}
\end{figure}
Legislators globally, are starting to codify their positions on proof of work as a technology (including Bitcoin). The USA is variously supporting or constricting the technology, according to \href{https://www.ncsl.org/research/financial-services-and-commerce/cryptocurrency-2021-legislation.aspx}{state legislatures}. Notably New York has \href{https://www.nysenate.gov/legislation/bills/2021/s6486}{submitted a bill} to ban mining for 3 years, while rust and farm belt states with energy build-out problems are \href{https://financialpost.com/fp-finance/cryptocurrency/texas-governor-abbott-turns-to-bitcoin-miners-to-bolster-the-grid-and-his-re-election}{providing incentives}. \par
The EU has just voted to add the whole of `crypto', including PoW, to the EU taxonomy for sustainable activities. This EU wide classification system provides investors with guidance as to the sustainability of a given technology, and can have a meaningful impact on the flows of investment. With that said the report and addition of PoW is not slated until 2025, and it is by no means clear what the analysis will be by that point. Meanwhile they're tightening controls of transactions, on which there will be more detail later.  
\subsection{Technical overview}
This section could be far more detailed, but this is pretty complex stuff. Instead, there's plenty of \href{https://github.com/bitcoinbook/bitcoinbook}{books and websites} that do a more thorough job, if the reader is interested. Each subsection will include a good external link where more depth can be found. This whistle stop tour of the main components of the protocol should provide enough grounding.\par
\subsubsection{ECDSA / SHA256 / secp256k1}
These technologies tend to use the same underpinning elliptic curve cryptography, and it makes sense to unpack this here just once, only in the context of Bitcoin, as this will be the main focus of our attention.\par
In Bitcoin the ECDSA algorithm is used on the \href{https://en.bitcoin.it/wiki/Secp256k1}{secp256k1} elliptic function to create a trapdoor. This (essentially) one way mathematical operation was originally the ``discrete log problem'' and part of the research in cryptography by Diffie and Hellman \cite{diffie1976new}. This is what binds the public and private keys in a key pair. In their mathematical construct a modulus operator creates an infinite number of possible variations on operations which multiply enormous exponential numbers together, in different orders, to create key pairs. In order to reverse back through the `trapdoor' a probably impossible number of guesses would have to be applied.\par
Latterly, elliptic curves such as the secp256k1 curve used in Bitcoin have substantially simplified the computation problems. Rather than exponentials used by Diffie Helmman instead a repeated operation is applied to an elliptic curve function, and this itself creates a discrete log problem trapdoor in maths, far more efficiently. Figure \ref{fig:ECDSA} suggests how this works. \par
This makes it easier, faster, and cheaper to provide secure key pairs on basic computational resources. Elliptic curve solutions are not `provably' secure in the same way as the Diffie-Hellman approach, and the security of this system is very sensitive to the randomness which is applied to the operation. Aficionados of Bitcoin use dice rolls or \href{https://www.hackster.io/news/alex-waltz-s-quantum-random-number-generator-for-bitcoin-uses-radioactive-decay-and-a-raspberry-pi-25a75316220f}{even more exotic} means to add entropy (randomness) when creating keys. This really isn't necessary, the software does this well enough.\par  
ECDSA has already been replaced by the more efficient Schnorr signature method \cite{schnorr1989efficient}, but this will take some time for organic adoption, and ECDSA will never be deprecated.\par
\begin{figure}
  \centering
    \includegraphics[width=\linewidth]{ECDSA}
  \caption{Given a start point on the curve and a number of reflection operations it's trivial to find a number at the end point, but impossible to find the number of \href{https://github.com/bitcoinbook/bitcoinbook/blob/develop/ch04.asciidoc}{hops} from the two end points alone. (CC Mastering Bitcoin second edition)}
  \label{fig:ECDSA}
\end{figure}
\subsubsection{Bitcoin script}
A Bitcoin script is a short chunk of code written into each transaction which gives conditions for the next spend. The \href{https://bitcoin.sipa.be/miniscript/}{limited scripting language} and the features built into wallets on top, allow for some clever additional options beside receiving and spending. In fact, some of the more innovative features such as discrete log contracts (detailed later) are quite powerful, and can interact with the outside world. Scripts allow spends to be contingent on multiple sets of authorising keys, time locks into the future, or both.\par
\subsubsection{Addresses \& UTXOs}
Ethereum has addresses which transactions flow in and out of. This is synonymous to a bank account number and so makes intuitive sense to users of banks. This is not the case in Bitcoin.\par
Bitcoin is a UTXO model blockchain. UTXO stands for unspent transaction output, and these are `portions' of Bitcoin created and destroyed as value changes hands (through the action of cryptographic keys). They are the basis of the evolving ledger. This process is described well by Rajarshi Maitra in \href{https://medium.com/bitbees/what-the-heck-is-utxo-ca68f2651819}{this post}.\par
\textit{``Every Transaction input consists of a pointer and an unlocking key. The pointer points back to a previous transaction output. And the key is used to unlock the previous output it points to. Every time an output is successfully unlocked by an input, it is marked inside the blockchain database as `spent'. Thus you can think of a transaction as an abstract “action” that defines unlocking some previous outputs, and creating new outputs.\\
These new outputs can again be referred by a new transaction input. A UTXO or `Unspent Transaction Output' is simply all those outputs, which are yet to be unlocked by an input.\\
Once an output is unlocked, imagine they are removed from circulating supply and new outputs take their place. Thus the sum of the value of unlocked outputs will be always equal to the sum of values of newly created outputs (ignoring transaction fees for now) and the total circulating supply of bitcoins remains constant.''}\par
Fresh UTXOs are created as coinbase transactions, rewarded to miners who successfully mine a block. These can be spent to multiple output as normal. This is how the supply increases over time.
\subsubsection{Halving}
As mentioned eariler, roughly every four year the `block reward' given to miners halves. This gives the issuances schedule of Bitcoin; \href{http://bashco.github.io/Bitcoin_Monetary_Inflation/}{it's monetary inflation}. This `controlled supply' feature was added to emulate the growth of physical asset stocks through mining. It's exhaustively \href{https://en.bitcoin.it/wiki/Controlled_supply}{explained elsewhere} and is somewhat immaterial to our transactional use case in metaverse applications.
\subsubsection{Difficulty adjustment}
The difficult adjustment (also mentioned earlier) shifts the difficulty of the mining algorithm globally to re-target one new block every 10 minutes. This means that adding a glut of new mining equipment will increase the issuance of Bitcoins, in favour of the new mining entity, for up to 2 weeks, at which point the difficulty increases, the schedule resets, and the advantage to the new miner is diffused. Equally this protects the network against significant loss of global mining hashrate, as happened when China comprehensively banned mining. Again, this is explained in \href{https://en.bitcoin.it/wiki/Difficulty}{more detail} elsewhere.
\subsubsection{Bitcoin nodes }
The Bitcoin network can be considered a triumvirate of economic actors, each with different incentives. These are:
\begin{itemize}
\item Holders and users of the tokens, including exchanges and market makers, who make money speculating, \href{https://en.wikipedia.org/wiki/Arbitrage}{arbitraging}, and providing liquidity into the network. Increasingly this is also real `money' users of BTC, earning and spending in pools of circular economic activity. Perversely Bitcoin as a money is the fringe use case at this time.
\item Miners, who profit from creation of new UTXOs, and receive payments for adding transactions to the chain. In return they secure the network by validating the other miners blocks according the the rules enforced by the node operators.
\item Node operators, \href{https://www.truthcoin.info/blog/measuring-decentralization/}{who enforce the consensus} ruleset which the miners must abide by in order to propagate new transaction into the network. In return node operators optimise their trust minimisation, and help protect the network from changes which might undermine their speculation and use of the tokens \cite{blocksizewars}.
\end{itemize}
There are currently around \href{https://bitnodes.io/}{15,000 Bitcoin nodes} distributed across the world. 
Since IT engineer \href{https://stadicus.com/}{Stadicus} released his \href{https://raspibolt.org/backstory.html}{Raspibolt guide} in 2017 there has been an explosion of small scale Bitcoin and Lightning node operators. 
Around thirty thousand Raspberry Pi Lightning nodes (which are also by definition Bitcoin nodes) run one of a big selection of \href{https://github.com/bavarianledger/bitcoin-nodes}{open source distributions}, with the most noteworthy explained alongside their stand-out feature-set:
\begin{itemize}
\item \href{https://github.com/rootzoll/raspiblitz}{Raspiblitz} offers fully opensource lightning focused functionality with a touchscreen display
\item \href{https://github.com/mynodebtc/mynode}{Mynode} focuses on easy of use through a web interface and has many modules which users can try out.
\item \href{https://github.com/getumbrel/umbrel-os}{Umbrel} is a more user friendly multi purpose node allowing access to a suite of Bitcoin and self sovereign individual tools.
\item \href{https://wiki.ronindojo.io/}{RoninDojo} is designed for use alongside the privacy focused \href{https://samouraiwallet.com/}{Samourai mobile wallet}.
\item \href{https://github.com/fort-nix/nix-bitcoin}{nix-bitcoin} focuses on security of the underlying operating system by building on NixOS.
\item \href{https://nodl.it}{NODL} is a premium prebuilt node focusing on security of the more performant hardware, and underlying operating system. It offers additional privacy tools.
\item \href{https://store.start9labs.com/collections/embassy}{Start9 Embassy} is a small form factor prebuilt unit at a lower price. It is a venture capital funded project with a more restrictive license but offers a suite of easy to use self sovereignty tools including Bitcoin. 
\item \href{https://www.osinteditor.com/NXN/mp_march28_f3/index.html}{The CLBoss plugin} can be combined with may of the above under `Core Lightning' to automate much of the operations and is a contender for our deployment too.
\end{itemize}
This is as good a time as any to start shaping a mission statement: \textbf{This book builds toward a new metaverse focused suite of tools, that utilises the nix-bitcoin distribution for the Bitcoin node}. There's a lot of components that haven't been covered yet, but nix-bitcoin is the first selected piece of open source software.
%\href{https://nydig.com/research/nydig-bitcoin-101}{Consider  mining the NYDIG primer for information}

\subsubsection{Wallets, seeds, keys and BIP39}
In all the cryptographic systems described in this book everything is derived from a private key. This is an enormous number, and the input to the trapdoor function described earlier. As usual, it's beyond the scope of this book to `rehash' the detail. Prof Bill Buchanan OBE has a \href{https://medium.com/asecuritysite-when-bob-met-alice/can-i-derive-the-private-key-from-the-public-key-ba3609256ec}{great post} on the basic version of this process.\par
In modern wallets, private keys (and so too their public keys), and addresses, are generated hierarchically. This is all part of \href{https://github.com/bitcoin/bips/blob/master/bip-0032.mediawiki}{BIP-0032}. It starts with a single \href{https://www.wolframalpha.com/input?i=2\%5E512}{monstrously large} number of up to 512 bits. From this are crafted Hierarchical Deterministic (HD) wallets, which use `derivation paths' to make a tree of public/private key pairs, all seeded from this first number. This means that knowing the initial number, and the derivation path applied to it (just another number), wallets can search down the tree of derivations and find all the possible addresses. In this way a whole group of active addresses belonging to an entity can be held conveniently in one huge number (a concatenation of the input and path). This is the seed. Seeds are even more conveniently abstracted into a mnemonic called a seed phrase. Anyone interacting with these systems will see a 12 word (128 bits of entropy which is considered to be \href{https://twitter.com/adam3us/status/1433375602808066049}{`enough'}) or 24 word (256 bit) seed phrase. That phrase accesses the whole of the assets stored by that entity in the blockchain under it. A master key. These seeds can be \href{https://vault12.com/securemycrypto/cryptocurrency-security-how-to/dice-crypto-recovery-seed/}{generated by hand} with dice, remember it's just a huge number and the onward cryptography at play here.\par
An address in Bitcoin is derived from the public/private key pair. Again this is a one way hash function. The public/private keys can't be found from the address. Addresses are really only a thing in wallets. They contain the element necessary to interact with the UTXOs. Many UTXOs can reside under an address, in that they just share the same keys. Wallets and nodes can monitor the blockchain to see transactions that `belong' to addresses owned by the wallet, then they can perform unlocking of those funds to move them, through operations on the UTXOs via keys.\par
It's beyond the scope of this book to review or suggest software in detail, but \href{https://bluewallet.io/}{Bluewallet} on mobile devices, and \href{https://sparrowwallet.com/}{Sparrow Wallet} on desktop devices provide rich basic functionality if a reader wishes to get started immediately. Note that these software wallets send your extended public key (the path of those keys) to the wallet providers server, for the monitoring of the blockchain to happen on it's behalf. They're updated by the software vendor, not the blockchain direct. To get this to `privacy best practice' commensurate with the aim of this book it's necessary to run a full node as detailed above, and connect the wallet software to that on a secure or local connection. This is what we will build to over the course of the book, but for our special metaverse purpose. 
\subsection{Upgrade roadmap}
\subsubsection{Taproot}
`Taproot' is the most recent upgrade to the Bitcoin network. It was first \href{https://lists.linuxfoundation.org/pipermail/bitcoin-dev/2018-January/015614.html}{described in 2018} on bitcoin-dev mailing list, and become \href{https://github.com/bitcoin/bips/blob/master/bip-0341.mediawiki}{BIP-0341} in 2019. It brings improved scripting, smart contract capability, privacy, and Schnorr signatures \cite{schnorr1989efficient}, which are a maximally efficient signature verification method. The network will always support older address types. It is rare to get such a large update to the network, and deployment and upgrade was carefully managed over several months under BIP-0008. Uptake will be slow as wallet manufacturers and exchanges add the feature. It can be considered an \href{https://transactionfee.info/charts/transactions-spending-taproot/}{upgrade in progress (0.3\%)}. Aaron van Wirdum, a journalist and educator in the space describes Taproot in detail in \href{https://bitcoinmagazine.com/technical/taproot-coming-what-it-and-how-it-will-benefit-bitcoin}{an article}.\par
\subsubsection{AnyPrevOut}
\href{https://anyprevout.xyz}{BIP-0118}, is a ``\href{https://en.bitcoin.it/wiki/Softfork}{soft-fork} that allows a transaction to be signed without reference to any specific previous output''. It enables ``Eltoo, a protocol that fulfils Satoshi's vision for nSequence''\par
This is Lightning Network upgrade technology in the main. The Eltoo \href{https://blockstream.com/eltoo.pdf}{whitepaper} or this more \href{https://fiatjaf.alhur.es/ffdfe772.html}{readable explanation} from developer fiatjaf go into detail.\par 
\subsubsection{CheckTemplateVerify}
\href{https://utxos.org/}{BIP-0119} is ``a simple proposal to power the next wave of Bitcoin adoption and applications. The underlying technology is carefully engineered to be simple to understand, easy to use, and safe to deploy''. At it's most basic it is a constructed set of output hashes, creating a Bitcoin address, which if used, can only be spent under certain defined conditions. This is a feature called `covenants'. It enables a feature called `vaults' which provides \href{https://github.com/jamesob/simple-ctv-vault/blob/7dd6c4ca25debb2140cdefb79b302c65d1b24937/README.md}{additional safety features} for custodians. There is currently \href{https://blog.bitmex.com/op_ctv-summer-softfork-shenanigans/}{some debate about the activation process}, and the feeling is that it won't happen (soon).
\subsubsection{Blind merge mining}
BIP-0301 allows `other' chains transactions to be mined into Bitcoin blocks, and for miners to take the fees for those different chains, without any additional work or thoughts by the miners. This is also a prerequisite for Drivechains (mentioned later), and improve Spacechains. In a way this can offer other chains the security model of the Bitcoin network, while increasing fees to miners, which might be increasingly important as the block subsidy falls. This is pretty fringe knowledge \href{https://bitcointalk.org/index.php?topic=1790.msg28696#msg28696}{originally proposed} by Satoshi, but has been refined since and is best explained by \href{https://www.youtube.com/watch?v=xweFaw69EyA}{Paul Sztorc elsewhere}. It is likely an important upgrade in light of the \href{https://www.truthcoin.info/blog/security-budget/}{security budget} of Bitcoin.
\subsubsection{Simplicity scripting language}
\href{https://blockstream.com/simplicity.pdf}{Simplicity} is a proposed contract scripting language which is \href{https://coq.inria.fr/}{`formally provable'}. This would provide a radical upgrade to confidence in smart contract creation. It is \href{https://github.com/ElementsProject/simplicity/blob/pdf/Simplicity-TR.pdf}{work in progress}, and looks to be incredibly difficult to develop in, despite the name. It is more akin to \href{https://en.wikipedia.org/wiki/Assembly_language}{assembly language}. Development has recently slowed, and the proposal requires a soft fork to Bitcoin. The main reason to think it stands a chance of completion vs other \href{https://lists.linuxfoundation.org/pipermail/bitcoin-dev/2022-March/020036.html}{similar proposals} is the powerful backing of \href{https://blockstream.com/}{Blockstream}, one of the main drivers of the Bitcoin ecosystem, run by Adam Back (potential co-creator of Bitcoin). 
\subsubsection{Ossification}
The Bitcoin code is aiming toward so called \href{https://en.wikipedia.org/wiki/Protocol_ossification}{``ossification''}. The complete cessation of development of the feature set. This would provide higher confidence in the protocol moving forward, as long term investors would be somewhat assured that the parameters of the technology would not change, and potentially pressure on the developers would reduce. There's a push to get some or all of the features described above in over the next few year before this happens. As ever this is a controversial topic within the development community. Notably Paul Sztorc, inventor of Drivechain feels strongly that cessation of innovation is a fundamental mistake.
\section{Extending the BTC ecosystem }
The following section are by no means an exhaustive view of development on the Bitcoin network, but it does highlight some potentially useful ideas for supporting metaverse interactions in a useful timeframe.
\subsection{Block \& SpiralBTC}
Block (formally the payment processor ``Square'' is now an umbrella company for several smaller 'building block' companies, all of which are major players in the space. \par
SpiralBTC, formally `Square Crypto' (a subsidiary of Square) is funding development in Bitcoin and Lightning. Their main internal product is the \href{https://spiral.xyz/blog/what-were-building-lightning-development-kit/}{Lightning Development Kit} (LDK). This promising open source library and API will allow developer to add lightning functionality to apps and wallets. It is a useful contender for our metaverse applications. They also fund external open source development.\par
\subsection{BTCPayServer}
BTCPayServer is one of the recipients of a Spiral grant. It is a self hosted Bitcoin and Lightning payment processor system which allows merchants, online, and physical stores and businesses to integrate Bitcoin into their accounting systems. It might seem that if one were to use Bitcoin then a simple address published on a website might be enough, but this is far from privacy best practice. Using a single address creates a data point which allows external observers to tie all interactions with a given point of sale to all of the customers, and onward to all of their other transactions through the public ledger. Since we seek to employ cyber security best practice will will avoid \href{https://en.bitcoin.it/wiki/Address_reuse}{the issues with address reuse}. Each Bitcoin address should be used just once. This is fine as there's essentially an \href{https://privacypros.io/btc-faq/how-many-btc-addresses}{unlimited number} of address.\par
In a metaverse application there is no website to interact with, but fortunately BTCPayServer is completely open source and extensible, has a strong support community, \href{https://docs.btcpayserver.org/API/Greenfield/v1/#operation/Invoices_CreateInvoice}{and an API} which could be integrated with a virtual world application. 
BTCPayServer supports the \href{https://docs.btcpayserver.org/LightningNetwork/}{main three} distributions of Lightning but would potentially need extending in order to work with newer technology like RGB or Omnibolt.
\section{Lightning (Layer 2)}
Lightning was a 2016 proposal by Poon and Dryja \cite{poon2016bitcoin}, and is a method for networks of channels of Bitcoin between parties, which can transfer value. The main public network is a community driven liquidity pool which enables scaling and speed improvements for the Bitcoin network. As with Bitcoin base chain there are multiple standards and approaches, but within Lightning these are not necessarily cross compatible with one another, resulting in severel Lightning networks. This is to our advantage as innovation is possible within these smaller networks. It is mainly `powered' by \href{https://plebnet.wiki/wiki/Main_Page}{thousands of volunteers} who invest in hardware and lock up their Bitcoin in their nodes, to facilitate peer-to-peer transactions. Zebka et al. found that although the network is ``fairly decentralised'' it is more recently skewing to larger more established nodes \cite{zabka2022short}. Though this is a grassroots technology the nature of the design means it can likely be trusted for small scale commercial applications.\par
The following text is from \href{https://medium.com/@johncantrell97?p=5cc72f2c664}{John Cantrell}, an engineer who works on Lightning.\par

\textit{``The Lightning Network is a p2p network of payment channels. A payment channel is a contract between two people where they commit funds using a single onchain tx.  Once the funds are committed they can make an unlimited amount of instant \& free payments over the channel.
You can think of it as a tab where each person tracks how much money they are owed.  Each time a payment is made over the channel both parties update their record of how much money each person has.  These updates all happen off-chain and only the parties involved know about them. When it`s time to settle up the two parties can take the final balances of the channel and create a channel closing transaction that will be broadcast on chain.  This closing transaction sends each party the final amounts they are owed. This means for the cost of two on-chain transactions (the opening and closing of the channel) two parties can transact an unlimited number of times and the overall cost of each transaction approaches zero with every additional transaction they make over the channel. Payment channels are a great solution for two parties to transact quickly and cheaply but what if we want to be able to send money to anyone in the world quickly and cheaply?  This is where the Lightning Network comes into play, it`s a p2p network of these payment channels. This means if Alice has a payment channel with Bob and Bob has a channel with Charlie that Alice can send a payment to Charlie with Bob`s help. This idea can be extended such that you can route a payment over an arbitrary number of channels until you can reach the entire world. Routing a payment over multiple channels uses a specific contract called a Hash Time Locked Contract (HTLC).  It introduces the ability for Bob and any other nodes you route through to charge a small fee.  These fees are typically orders of magnitude smaller than onchain fees. This all sounds great but what if someone tries to cheat? I thought the whole point of Bitcoin was that we no longer had to trust anyone and it sure sounds like there must be some trust in our channel partners to use the Lightning Network? The contracts used in Lightning are built to prevent fraud while requiring no trust.  There is a built-in penalty mechanism where if someone tries to cheat and is caught then they lose all of their money.  This does mean you need to be monitoring the chain for fraud attempts.''}

%\href{https://twitter.com/marcrjandrew/status/1478052587387568130}{Five facts about lightning}
Lightning is a key scaling innovation in the bitcoin network at this time. It is seeing rapid development and adoption (Figure \ref{fig:lightningAdoption}). The popular payment app ``Cash App'' integrates the technology, and `Lightning Strike' services the USA, El Salvador, and Argentina with zero exchange and tranmission fees.

\begin{figure}
  \centering
    \includegraphics[width=\linewidth]{lightningAdoption}
  \caption{\href{https://www.research.arcane.no/the-state-of-lightning}{Arcane research lightning adoption overview}.}
  \label{fig:lightningAdoption}
\end{figure}
It allows for unbound scaling of transactions (millions of transations per second compared for instance to around 45,000 TPS in the VISA settlement network). Transaction costs are incredibly low, and the transaction speed virtually instantaneous.\par
The most popular lightning software is \href{https://github.com/lightningnetwork/lnd#readme}{LND} from Lightning Labs or \href{https://github.com/ElementsProject/lightning}{C-Lightning} from Blockstream. The software can be run on top of any Bitcoin full node, in a browser extension with a limited node, in a mobile app as a client or a server, or a hybrid such as the Greenlight server \href{https://medium.com/breez-technology/get-ready-for-a-fresh-breez-multiple-apps-one-node-optimal-ux-519c4daf2536}{used by Breez wallet}. Different trust implications flow from these choices.
\subsection{Micropayments}
Possibly the most important affordance of the Lightning network is the concept of micropayments, and streaming micropayments. It is very simple to transfer even \href{https://satsymbol.com/}{one satoshi} on Lightning, which is one hundred millionth of a bitcoin, and a small fraction of a penny. This can be a single payment, for a very small goods or service, or a recurring payment on any cadence. This enables streaming payments for any service, or for remittance, or remuneration. These use cases likely have enormous consequences which are just beginning to be explored. Integration of this capability into metaverse applications will be explored later.
\subsection{BOLT12 and recurring payments}
\href{https://bolt12.org/}{BOLT12} is a new and developing 'standard' which simplifies and extends the capability of the network for recurring payments.

\subsection{LNBits}
LNBits is an open source, extensible, Lightning `source' management suite. It is self hosted, and can connect to a variety of Lightning wallets, further abstracting the liquidity to provide additional functionality to network users. Remember that all of these tools run without a third party, on a £200 setup, hosted at home or within a business. The best way to explore this is to describe \textit{some} of the plugins. 
\begin{itemize}
\item ``\href{https://github.com/lnbits/lnbits-legend#lnbits-v03-beta-free-and-open-source-lightning-network-walletaccounts-system}{Accounts System}; Create multiple accounts/wallets. Run for yourself, friends/family, or the whole world!''
\item \href{https://github.com/lnbits/lnbits-legend/tree/quart/lnbits/extensions/events#events}{Events plugin} allows QR code tickets to be created for an event, and for payments to be taken for the tickets.
\item \href{https://github.com/lnbits/lnbits-legend/tree/quart/lnbits/extensions/jukebox#jukebox}{Jukebox} creates a Spotify based jukebox which can be deployed online or in physical locations.
\item \href{https://github.com/lnbits/lnbits-legend/tree/quart/lnbits/extensions/livestream#dj-livestream}{Livestream} provides an interface for online live DJ sets to receive real-time Lightning tips, which can be split automatically in real-time with the music producer.
\item \href{https://github.com/lnbits/lnbits-legend/tree/quart/lnbits/extensions/tpos#tpos}{TPoS}, \href{https://github.com/arcbtc/LNURLPoS#lnurlpos}{LNURLPoS} \& \href{https://github.com/lnbits/lnbits-legend/tree/quart/lnbits/extensions/watchonly#watch-only-wallet}{OfflineShop} support online \href{https://rapaygo.com/}{and offline} point of sale (Figure \ref{fig:LnBitsPoS}).
\item \href{https://github.com/lnbits/lnbits-legend/tree/quart/lnbits/extensions/paywall#paywall}{Paywall} creates web access control for content. 
\item \href{https://github.com/LightningTipBot/LightningTipBot#lightningtipbot-}{LightningTipBot} is a custodial Lightning wallet and tip handling bot within the popular on Telegram instant messenger service.
\end{itemize}
\begin{figure}
  \centering
    \includegraphics[width=\linewidth*\real{0.8}]{LnBitsPoS}
  \caption{Two of the many \href{https://rapaygo.com/}{prebuilt} and \href{https://github.com/arcbtc/LNURLPoS}{kit} options for Lightning `point of sale'}
  \label{fig:LnBitsPoS}
\end{figure}
Together these plugins are incredibly useful primitives which are likely to be translatable to a multi party metaverse application. A proposal for building a more specific plugin along these lines is detailed later.\par
\textbf{LnBits is capable of backing every object in a metaverse scene as an economic actor, with a key which is compatible with Nostr. This makes it the best choice and it will likely form the core of the proposed metaverse stack.}
\subsection{Etleneum}
Etleneum is a centralised smart contract platform built around Lightning invoices. It is most notable as a sign of things to come. There are \href{https://etleneum.com/#/contracts}{many small contracts} available to try on the site, such as a \href{https://etleneum.com/#/contract/c8w0c13v75}{simple market} for moving value between lightning and Bitcoin layer 1, or this \href{https://simple-auction.etleneum.com/}{simple auction}.  Contracts are able to operate on data drawn from the wider web, and automatically send and receive lightning payments based on conditional states. It should be viewed as an experiment which allows tinkering in smart contracts, and therefore potentially useful for the software proposed in the final section. There are \href{https://notgeld.medium.com/lightning-network-computation-layer-27c7ba81a214}{suggestions} that this approach might (with some work) allow layer 3 computation more like Ethereum etc.
\subsection{Message passing}
It is possible to pass data alongside lightning payments, routing messages between parties across the global network. This means that a host of other applications can inherit the privacy and censorship resistance of the Lightning network. First amongst these has been simple message passing and group messenger clients such as Sphinx and Juggernaut. To be clear, this is considered by some to be a misappropriation of the function of the network. Once more developed use case has been demonstrated by the Impervious development team; they use the message passing capability to negotiate a virtual private network between two parties, using open source software. This allows a secure side channel between internet IP addresses to be opened without a trusted third party. This in itself is a much sought after function of privacy minded networking, and the basis for much of their Impervious browser feature set. 
\section{Liquid federation (layer 2)}
Liquid is an implementation on Blockstream \href{https://elementsproject.org/}{Elements}, and is itself part of the open source development contribution of Blockstream, the company started by Adam Back (of proof of work fame) and nearly a dozen other early cypherpunks and luminaries.\par 
The Liquid side chain network, and it's own attendant Lightning layer 2, is a fork of Bitcoin with different network parameters. In liquid the user of the network `pegs' into the Bitcoin network, swapping tokens out from BTC to L-BTC (this can of course mean very small subunits of 1 Bitcoin). Once tokens have been `locked' and swapped to Liquid the different network parameters used in the fork allow a different trust/performance trade-off. Liquid is fast on the L1 chain, cheaper to use at this time, and more private. The consensus achieved on this side chain network is faster because it is a far smaller group of node operators. The next block to be written to the side chain is chosen by a node operated by a member of a federation of dozens of major contributors to the Bitcoin technology space. These `trusted' nodes all check one another's security and network operations, meaning that the network is as secure as the aggregate of the trust placed in half of the membership at any one time. There are
\href{https://bitcoinmagazine.com/business/bitcoin-liquid-network-gains-six-new-federation-members}{still dozens} of major companies, development teams, and individual actors, with significant reputational investment.\par
``Federation members contribute to the Liquid Network's security, gain voting rights in the board election and membership process, and provide valuable input on the development of new features. Members also benefit from the ability to perform a peg-out without a third party, allowing their users to convert between L-BTC and BTC seamlessly within their platform.''\par
Crucially for our purposes here Liquid allows tokenised asset transfer. Anyone \href{https://docs.blockstream.com/liquid/developer-guide/developer-guide-index.html#issued-assets}{can issue} an asset on Liquid. Such transfers of assets may be orders of magnitude cheaper than on chain Bitcoin transactions, but still potentially orders of magnitude more expensive than a simple Lightning transaction of value on the Bitcoin network. \par 
Blockstream plan to add arbitrary (user generated) token support to their `Core Lightning' implementation at some point. This would be a very strong choice for specific use cases within an economically enabled metaverse application. When participants wish to `cash out' of the Liquid network they must do this through one of the federation members who activate the other side of the `two-way peg', dispensing the equivalent amount of Bitcoin. This is transparently handled through Blockstream's ``green wallet''.\par
All of this has the advantage of a far lower energy footprint compared to the main chain, but it's not quite ready with a full suite of affordances. \par
The Liquid network is being used as the underlying asset for a novel new global financial product. El Salvador are working with Blockstream to issue a nation state backed bond. 
\section{Bitcoin Layer 3}
Increasingly important features of modern blockchain implementations are programmability through smart contracts, and issuance of arbitrary tokens. Assigning a transaction to represent another thing like an economic unit, energy unit, or real world object, and operating on those abstractions within the chain logic. Chief among these use cases are stablecoins such as Tether, which are pegged to national currencies and described in the next section. Bitcoin has always supported very limited contracts called scripts, and stablecoin issuance has existed in Bitcoin since 
\href{https://www.omnilayer.org/}{Omni Layer}. Omni was the first issuer of Tether, but more recently these important features have passed to other layer one chains. This year is likely to see the \href{https://www.hiro.so/blog/bitcoin-ecosystem-a-guide-to-programming-languages-for-bitcoin-smart-contracts}{resurgence of this capability} on Bitcoin, which of course benefits from a better security model. Once again, there is a stong assertion by some that \href{https://lists.linuxfoundation.org/pipermail/bitcoin-dev/2022-April/020227.html}{this isn't even possible}. The debate is complex and unresolved.\par 
In order to properly understand the use of Bitcoin based technologies in metaverse applications it is necessary to examine what these newer `layer 3' ideas might bring. 
\subsection{LNP/BP and RGB}
\href{https://giacomozucco.com/layers-before-bitcoin}{LNP/BP} is a non profit standards organisation in Switzerland which contributes to open source development of Bitcoin layer 3 solutions into the Lightning protocol, and Bitcoin protocol (LNP/BP). One of the core product developments within their work is the `RGB' protocol, which is somewhat of a meaningless name, evolved from ``coloured coins'' which were an early tokenised asset system on the Bitcoin network. RGB represents red, green, and blue. The proposal is built upon research by \href{https://petertodd.org/2016/commitments-and-single-use-seals}{Todd} and \href{https://giacomozucco.com/#intro}{Zucco}. RGB is regarded as arcane Bitcoin technology, even within the already rarefied Bitcoin developer communities. Zucco provides the \href{https://bitcoinmagazine.com/culture/video-interview-giacomo-zucco-rgb-tokens-built-bitcoin}{following explanation}: \par
\textit{``When I want to send you a bitcoin, I will sign the transaction, I will give the transaction only to you, you will be the only one verifying, and then we’ll take a commitment to this transaction and that I will give only the commitment to miners. Miners will basically build a blockchain of commitments, but without the actual validation part. That will be only left to you. And when you want to send the assets to somebody else, you will pass your signature, plus my signature, plus the previous signature, and so on.''}\par
This is non-intuitive explanation of Todds `single-use-seals', applied to Bitcoin, with the purpose of underpinning arbitrary asset transfer secured by the Bitcoin network. In this model the transacting parties are the exclusive holders of the information about what the object they are transferring actually represents. This primitive can (and has) been expanded by the LNP/BP group into a concept called `client side validation'. 
It's appropriate to explain this concept several times from different perspectives, because this is potentially a profoundly useful technology for metaverse applications.\par
\begin{itemize}
\item A promise is made to spend a multi output transaction in the future. This establishes the RGB relationships between the parties.
\item One of the pubkeys to be spent to is known by both parties.
\item The second output is unknown and is a combination of the hash of the state, and schema, from the operation which has been performed.
\item When the UTXO is spent the second spends pubkey can be processed against the shared data blob to validate the shared state in a two party consensus
\item This is now tethered to the main chain. Some tokens from the issuance have gone to the recipiant, and the remainder have gone back to the issuer. More tokens can be issued in the same way from this pool. 
\item A token schema in the blob will show the agreed issuance and the history back to the genesis for the token holder. 
\item The data blob contains the schema which is the key to RGB functions and the bulk of the work and innovation. 
\item Each issuance must be verified on chain by the receiving party. 
\end{itemize} 
This leverages the single-use-seal concept to add in smart contracts, and more advanced concepts to Bitcoin. Crucially, this is not conceptually the same as the highly expressive `layer one' chains which offer this functionality within their chain logic. In those systems there is a globally available shared consensus of `state'. In the LNP/BP technologies the state data is owned, controlled, and stored by the transacting parties. Bitcoin provides the crytographic external proof of a state change in the event of a proof being required. This is an elegant solution in that it takes up virtually no space on the blockchain, is private by design, and is extensible to layer 2 protocols like Lightning.\par
This expanding ecosystem of client side verified proposals is as follows:
\begin{itemize}
\item RGB smart contracts
\item RGB assets are fungible tokens on Bitcoin L1 and L2, and non fungible Bitcoin L1 (and somewhat on L2).
\item Bifrost is an \href{https://github.com/LNP-BP/presentations/blob/master/Presentation slides/Bifrost.pdf}{extension} to the Lightning protocol, with it's own Rust based node implementation, and backwards compatibility with other nodes in the network. This means it can transparently participate in normal Lightning routing behaviour with other peers. Crucially however it can also negotiate passing the additional data for token transfer between two or more contiguous Bifrost enabled parties. This can be considered an additional network liquidity problem on top of Lightning, and is the essence of the ``Layer 3'' moniker associated with LNP/BP. It will require a great number of such nodes to successfully launch token transfer on Lightning. As a byproduct of it's more `protocol' minded design decisions Bifrost can also act as a generic peer-to-peer data network, enabling features like Storm file storage and Prometheus.
\item \href{https://www.aluvm.org/}{AluVM} is a RISC based virtual machine (programmable strictly in assembly) which can execute Turing complete complex logic, but only outputs a boolean result which is compliant with the rest of the client side validation system. In this way a true or false can be returned into Bitcoin based logic, but be arbitrarily complex within the execution by the contract parties.
\item Contractum is the proposed smart contract language which will compile the RGB20 contracts within AluVM (or other client side VMs) to provide accessible layer 3 smart contracts on Bitcoin. It is a very early proposal at this stage.
\item  Internet2: ``Tor/noise-protocol Internet apps based on Lightning secure messaging
\item Storm is a lightly specified escrow-based bitcoin data storage layer compliant with Lightning through Bifrost.
\item Prometheus is a lightly specified multiparty high-load computing framework.
\end{itemize}
Really, any compute problem can be considered applicable to client side validation. In simplest terms a conventional computational problem is solved, and the cryptographically verifiable proof of this action, is made available to the stakeholders, on the Bitcoin ledger.\par 
Less prosaically, at this stage of the project the more imminent proposed affordances of LNP/BP are described in `schema' \href{https://github.com/LNP-BP/LNPBPs}{on the project github}. The most interesting to the technically minded layperson are:
\begin{itemize}
\item \href{https://github.com/LNP-BP/LNPBPs/blob/master/lnpbp-0020.md}{RGB20} fungible assets. This could be stablecoins like dollar or pounds representation. This is a huge application area for Bitcoin, and similar to Omni, which will also be covered next.
\item \href{https://github.com/LNP-BP/LNPBPs/blob/master/lnpbp-0021.md}{RGB21} for nonfungible tokens and ownership rights. In principle BiFrost allows these to be transferred over a future version of the Lightning network, significantly lowering the barrier to entry for this whole technology. This is slated for release later in the year.
\item \href{https://github.com/LNP-BP/LNPBPs/issues/29}{RGB22} may provide a route to identity proofs. This is covered in detail later.
\end{itemize}
Federico Tenga is CEO of `Chainside' and an educator and consultant in the space. He has written an up-to-date \href{https://medium.com/@FedericoTenga/understanding-rgb-protocol-7dc7819d3059}{``primer''}, which is still extremely complex for the uninitiated, but does capture how the RGB token transfer system works. That medium article also touches on Taro, which is next.
\subsection{Taro}
Taro is an very new \href{https://lightning.engineering/posts/2022-4-5-taro-launch/}{initiative by Lightning Labs} to allow assets to transmit on the Lightning network. It is more similar to RGB above than Omnibolt below. They say: \textit{``Taro enables bitcoin to serve as a protocol of value by allowing app developers to integrate assets alongside BTC in apps both on-chain and over Lightning. This expands the reach of Lightning Network as a whole, bringing more users to the network who will drive more volume and liquidity in bitcoin, and allowing people to easily transfer fiat for bitcoin in their apps. More network volume means more routing fees for node operators, who will see the benefits of a multi-asset Lightning Network without needing to support any additional assets.''}\par
The project has clearly been \href{https://github.com/roasbeef/bips/tree/bip-taro}{under development} by the lead developer at Lightning Labs for some years and seems both \href{https://lightninglabs.substack.com/p/bitcoinizing-the-dollar-and-the-world?s=r}{capable} and mature, though they are obviously following the model of `co-opting' open source ideas (from RGB) to garner venture capital funding. They \href{https://github.com/bitcoin/bips/pull/1298/commits/4daba8c373c777defb48136795382803c137502c}{credit RGB} in the github. More will doubtless be added to this section and it seems a contender for our metaverse purposes, though somewhat less capable than RGB upon which it's based.
\subsection{Slashpay}
Slashpay is a very promising and recent product, and part of a suite of interlinked (Bitcoin compatible) layer 3 ideas. It is \textit{not} a blockchain technology, but it is highly intersectional with Bitcoin. The Synonym suite is advocating what they call the `Atomic Economy', an overarching abstraction of any of the technologies in the space into a single cryptographic key pair, with all the other products nested under it. The suite will be selected and unpacked for metaverse purposes later, but for this section the Slashpay product integrates as a layer three idea, building upon lightning. \par
Slashpay gathers all of the available Lightning invoice and QR code `standards' under an abstracted metastandard with it's own key pair. This allows a negotiation between party and counterparty in code, which settles on agreeable standards for the value transfer. The Slashpay mediation metadata is embedded in the top level Slashtag QR code. Within the Slashpay element of the data is a priority list for the exchange, which can be changed by the user according to their capability and preferences.\par 
The code to support this is in a early stage. There is a minimum viable product which can negotiate between online Lightning nodes. The advantage of the Slashpay method, and the thing that puts it in the layer 3 section, is that in the event of a failed Lightning payment (for whatever reason), the software can then default to the next payment method in it's negotiated list. This would most likely be another (different) Lightning attempt, or a Bitcoin main chain transaction.\par
This technical ecosystem uses a `distributed hash table' to hold and negotiate keys, stored using Hypercore, another external project based on Bittorrent. This is the \href{https://hypercore-protocol.org/protocol/#hyperswarm}{`Hyperswarm DHT'}, which has potential uses elsewhere in the metaverse use cases.\par
There is clearly additional complexity in setting this system up, but once in place it might provide a unified way to scale capability under the evolving standard.  
\subsection{Spacechains}
Spacechains is a \href{https://medium.com/@RubenSomsen/21-million-bitcoins-to-rule-all-sidechains-the-perpetual-one-way-peg-96cb2f8ac302}{proposal} by Ruben Somsen. It is a way to provide the functionality of any conceivable blockchain, by making it a sidechain to Bitcoin. \par
Like RGB described earlier it's a it's a single use seal, but which can be closed by the highest bidder.\par
In a spacechain the Bitcoin tokens are destroyed in order to provably create the new spacechains tokens at a 1:1 value. These new tokens only have worth moving forward within the new chain ecosystem they represent, as they cannot be changed back. They nontheless have the same security guarantees as the bticoin main chain, though with a radically reduced ecological footprint (x1000?), and higher performance. Each `block' in the new chain is a single bitcoin transaction. The high level features are:\par
\begin{itemize}
\item Outsource mining to BTC with only a single tx per block on the main chain.
\item One way peg, Bitcoin in burnt to create spacechain tokens.
\item Allows permissionless chain creation, without a speculative asset.
\item Fee bidding BMM is space efficient and incentive compatible. Miner just take the highest fees as normal.
\item Paul Sztorc raised the idea
\item It's best with a soft fork but possible without
\end{itemize}

%\subsection{Drivechain}
%\lipsum[50]
%\subsection{Softchains}
%\lipsum[50] 
\subsection{Statechains, drivechain, softchains} 
There are many \href{https://gist.github.com/RubenSomsen/96505e99dc061d6af6b757ff74434e70}{proposals for layer 2 scaling solutions} for the bitcoin network. Ruben Somsen \href{https://gist.github.com/RubenSomsen/c9f0a92493e06b0e29acced61ca9f49a}{describes Softchains, Stateschains, and Spacechains}, while  \href{https://www.drivechain.info/literature/index.html}{Drivechain is described} by the author Paul Sztorc on the project web pages and is split across \href{https://github.com/bitcoin/bips/blob/master/bip-0300.mediawiki}{BIP-0300} for drivechain and \href{https://github.com/bitcoin/bips/blob/master/bip-0301.mediawiki}{BIP-0301} for a ``blind merge mining'', a soft fork which it's unlikely to get. They are all hypothetical with the exception of sidechains.  

\section{Other chains and networks}
It's useful to make some `honourable mentions' of other options as this technology is moving so fast. These chains are viewed by some as a kind of triage for ideas which might one day find their way into Bitcoin. This is potentially most true of zk rollups which might eventually migrate from privacy and scaling experiments on other chains. This list simply isn't very useful in terms of judging other chains, as they rise and fall so fast. To be clear, there is a fundamental \textit{legal} difference between all of the 15,000 or so attempts at layer 1 chains after Bitcoin, in that they are controlled by a subset of people who have an incentive to lie about the usefulness of the technologies they are invested in. This lack of useful decentralisation has been touched on but is dealt with in detail by Microstrategy CEO Michael Saylor in a \href{https://www.youtube.com/watch?v=mC43pZkpTec}{four hour podcast} with Ai researcher Lex Fridman. It's interesting that Saylor (who is a significant educator in the space) views custodial companies holding Bitcoin on behalf of their clients as Bitcoin layer 3. All of this is new enough that virtually off of it is contested by someone. To demonstrate the difference between Bitcoin as a property vs alt coins as `securities in law' it's useful to see the allocations of tokens to seed investors in some of the newer chains in Figure \ref{fig:messariICO}.

\begin{figure}
  \centering
    \includegraphics[width=\linewidth]{messariICO}
  \caption{Allocations given at the beginning of public blockchain, by Messari.}
  \label{fig:messariICO}
\end{figure}

\subsection{Layer 1 chains}

\begin{itemize}
\item Monero is a privacy focused coin with a great deal of credible and competent developer attention. It attracts a lot of criticism precisely because it's anonymity by design makes it perfect for nefarious activity such as drugs markets. Privacy advocates in the space say that total unaccountability of money is a `must have' feature and that \href{https://sethforprivacy.com/posts/dispelling-monero-fud/#introduction}{critiques of the chain} are in bad faith.
\item Solana is a far more centralised layer 1 proposition which uses a few hundred highly performant nodes to achieve high transaction throughput. The consensus algorithm is the novel \href{https://solana.com/solana-whitepaper.pdf}{``proof of history''} system. Development of the technology has been funded and supported by huge venture capital investment, and even though the chain is quite unreliable it seems that the vested interests of the investors can keep interest going. It is cheaper, and more useful than Bitcoin and Ethereum, but lacks longevity and reliability . 
\item Polkadot is much hyped within the ``crosschain'' protocol community. These chains connect the logic of a smart contract on one chain to that of another. In practice, while it is possible that this is useful for distributed finance products, it seems that chains such as DOT might be promising more than the markets actually want or need. Governance of the token is a DAO like model where staking (locking up) the tokens theoretically controls the direction of the product.
\item Terra is a relatively new \href{https://assets.website-files.com/611153e7af981472d8da199c/618b02d13e938ae1f8ad1e45_Terra_White_paper.pdf}{ecosystem offering} with a stablecoin, and DeFi built in. It is currently in ascendancy (\$15 in a year) and seems to have a stable and useful underpinning, but is far too new to judge with any certainty. There's more on this in the stablecoin section later.
\item Avalanche AVAX is a newer, `faster' and more eco friendly DeFi ecosystem which promises returns within it's own framework of permissionless money. It is one of the relative success stories of the DeFi narrative. It's unclear what the value proposition, and sustainability of this token actually are.
\item Tezos is an well established player with an early and somewhat battle tested proof of stake mechanism and distributed governance model. It has attracted many high profile partnerships and sponsors, but is primarily seeking to be a store of value token like bitcoin, which exposes the chain to the ``winner takes all'' landscape of digital money. There are some compelling NFT advocates of the technology, which is certainly `greener' and cheaper to use, but the longevity in such an irrational market is uncertainly because it does not seem to have the network effect and growth velocity. 
\item Algorand's ALGO token purports to be a more modern and useful proof of stake value transfer chain. It is fundamentally similar to Tezos.
\item IOTA is noteworthy, interesting, and established concept, with an edge use case. It is the `distributed ledger of the internet of things', the much hyped and clearly extant ecosystem of edge compute, sensors, smart devices etc. The marketing around IOTA correctly identifies it's positioning and potential within this developing technology ecosystem, but it's primary use case is too nascent and too niche to discuss in the same basket as the other ideas.
\item VeChain is a long established platform with significant industry adoption which still doesn't represent it's market capitalisation, usefulness, or future success. It is the most exposed of all the chains to the assertion that immutable global ledgers of real work asset tracking somehow protect from fraudulent behaviour. It's perhaps useful, but mainly in a highly automated industry 4.0 environment, with minimal human interaction.
\item Cardano foundation ADA is one of the more established players and has been developing methodically and slowly. They have made great strides in successfully enabling a provable proof of stake consensus structure. Proof of stake nontheless has significant problems in that tokens and therefore control inevitably concentrates over time. There is no proposed solution to this. They have working products and partnerships, but perhaps not as many as the market cap of the ecosystem would suggest. 
\item HBAR claims ``third generation'' blockchain technology, with carbon positive, high speed, distributed applications. There are always tradeoffs bound by physical constraints within distributed computing, and Hedera HBAR has been accused of a cryptographic model which is inherently insecure.
\item EOS is one of the early major successes of the ICO funding model in 2017, and they amassed an enormous warchest of bitcoin which they still hold. The onus is on them to deliver some kind of product, and they have the funds to do so.
\end{itemize}
\section{Risks and mitigations}
\subsection{Digital assets}
For digital assets more generally it is useful to look at the recent \href{https://www.whitehouse.gov/briefing-room/presidential-actions/2022/03/09/executive-order-on-ensuring-responsible-development-of-digital-assets/}{``whole government executive order''} signed by President Biden. It is mainly framed in terms of ``responsible innovation, and leadership'' in the new space, is a product of multi agency collaboration, and has been long anticipated. It identifies high level risks, aspirations, and challenges, and strongly hints toward development of a ``digital dollar'' (CBDC, expanded later). The risks sections show how legislators are framing this, so it's useful to break down here.\par
\begin{itemize}
\item Consumer and business protections. This is likely to pertain to custodians and is much needed. Misselling is rife. Security presents a challenge.  
\item Systemic risk, and market integrity are a concern. The legislators clearly worry about contagion risks from the sector.
\item Illicit finance (criminality and sanction busting etc) are a concern, but not particularly front and centre\cite{moser2013inquiry}. Criminality in 2021 was a mere 0.15\% of transactions according to Chainalysis, but this number varies year to year. The US treasury department has recently published a National Risk Assessments for Money Laundering, Terrorist Financing, and Proliferation Financing. This is a comprehensive report and speaks to careful research across the space. It is broken into \href{https://home.treasury.gov/news/press-releases/jy0619}{three parts}. Perhaps surprisingly, while they do see activity in these areas, they do not rate the risk as very significant. Cash remains the main problem for illicit funding. There is some talk that the nature of public blockchain analysis allows greater oversight of these tools and that this is to the advantage of government and civil enforcement agencies.
\item Highlighting the need for international coordination suggests they are mindful of jurisdictional arbitrage. 
The partial regulatory capture of these technologies, where activity flows to globally more lenient legislative regimes, continues to be a concern. Many of the centralised exchanges for instance are located in tax havens such as Malta. As the world catches up with these products it is likely that this will be smoothed out.
\item Climate goals, diversity, equality and inclusion are mentioned. It seems that the ``environment'' aspect of ESG is more important then ``social'' and ``governance'' at this time.
\item Privacy and human rights are mentioned.
\item Energy policy is highlighted, including grid management and reliability, energy efficiency incentives and standards, and sources of energy supply.
\end{itemize}

\subsection{Bitcoin specifically}
\noindent In addition it's useful for this document to focus more on the technical challenges to the Bitcoin network.\par
\begin{itemize}
\item The block reward is reduced every 4 years (epochs). This means a portion of the mining reward is trending to zero, and nobody knows what effect this will have on the incentives for \href{https://www.truthcoin.info/blog/security-budget-ii-mm/}{securing the network} through proof of work \cite{carlsten2016instability}.
\item Stablecoins are a vital transitional technology (described later) but do not meaningfully exist yet on the Bitcoin network. This may change.
\item Bitcoin lacks privacy by design. All transactions are publicly viewable. This is a major drag to the concept of BTC as a money. Upgrade of the network is possible, and has indeed been achieved for a Bitcoin fork called Litecoin \cite{fuchsbauer2019aggregate}. 
\item The Lightning network (described later) has terrible UX design at this time. 
\item The basic `usability' of the network is still poor in the main. Any problems which users experience demand a steep learning curve and risk loss of funds. There is obviously no technical support number people can call. 
\item Only around one billion unspent transactions can be generated a year on the network. This means that it might become impossible for everyone on the planet to have own their own Bitcoin address (with it's associated underpinning UTXO).  
\item Chip manufacture is concentrated in only a few companies and countries, as identified by Matthew Pines. He also identifies the \href{https://www.btcpolicy.org/#Research}{following points}
\item Potential constraints on monetary policy flexibility.
\item Future protocol changes.
\item Unanticipated effects the domestic and international energy system.
\item Vulnerability to adversary attacks.
\item Other unknown, unanticipated risks given Bitcoin’s limited 13-year history.
\end{itemize} 


%\subsection{Crime and santion busting}
%
