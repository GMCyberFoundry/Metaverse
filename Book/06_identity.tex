This is where an external contributor can add information about DID and SSI
\section{Some basics about distributed identity}
\subsection{DID principles}
persistent identifier (permanent)
Resolvable
cryptographically verifiable
decentralised
\subsection{primatives}
lifetime
portable
does not depend on third party
cannot be taken away
\subsection{What's in a DID document?}
it's to bootstrap the services
one or more public keys
One or more services which can be used
timestamps, sigs, proofs, delegations, authorisations
minimum amount of information
DID controller = DID subject
\subsection{What are the challenges?}
    fragmentations
    scaling to a world where the user is presented with thousands of signoffs
    how to visualise and abstract and glom this stuff
    How to care about the right level of information
\subsection{keystone uses}
\begin{itemize}
\item health
\item qualifications and certifications
\item finance
\item contacts
\item calendar
\end{itemize}  
\section{Technologies}
\subsection{Slashtags}
Slashtags is a new distributed identity open method being developed by Bitfinex and Tether under the Synonym suite. It uses bitcoin keys for authentication, but communicates a schema through a metadata exchange.
\subsection{LNURL-Auth}
\lipsum[50]
\subsection{Sovrin}
Sovrin foundation are established global players in distributed identity and we wish to apply to join their federation of credential issuers.
\subsection{Keri}
\lipsum[50]
\subsection{Atala Prism}
\lipsum[50]
\subsection{Microsoft ION}
While working at Microsoft on ION Daniel Buchner (now working at Square) or Henry Tsai \href{https://github.com/decentralized-identity/ion/blob/master/docs/Q-and-A.md}{said the following}, which is worth quoting verbatim:\\
``While ledger-based consensus systems, on the surface, would seem to provide the same general features as one another, there are a few key differences that make some more suitable for critical applications, like the decentralized identifiers of human beings. Some of these considerations and features are:
\begin{itemize}
\item The system must be open and permissionless, not a cabal of authorities who can exclude and remove participants.
\item The system must be well-tested, and proven secure against attack over a long enough duration to be confident in.
\item The system must produce a singular, independently verifiable record that is as immutable as possible, so that reversing the record the system produces is infeasible.
\item The system must be widely deployed, with nodes that span the globe, to ensure the record is persisted.
\item The system must be self-incentivized, so that nodes continue to operate, process, and secure the record over time. The value from operation must come from the system directly, because outside incentive reliance is itself a vector for attack.
\item The cost to attack the system through any game theoretically available means must be high enough that it is infeasible to attempt, and even if an ultra-capitalized attacker did, it would require a weaponized mobilization of force and resources that would be obvious, with options for mitigation.\\

The outcome:

\item Number 1 eliminates private and permissioned ledgers
\item Number 2 eliminates just about all other ledgers and blockchains, simply because they are inadequately tested
\item For the metrics detailed in 3-6, Bitcoin is so far beyond all other options, it isn't even close - Bitcoin is the most secure option by an absurdly large margin.''
\end{itemize}

On the surface then it might seem that the choice is Bitcoin again, and indeed that the open source Microsoft ION stack is a natural choice. 

\subsection{Atala Prism (ADA Ecosystem)}
\href{https://medium.com/coinmonks/part-2-using-nethereum-in-unity-5b09f2d8c718}{This Medium post} lists the pre-requisites required for interaction with Eth within Unity. At this stage this hasn't been expanded out. It's based around the .NET `nethereum' library which can be found on their \href{https://github.com/Nethereum/Nethereum}{github}.    
   
  
\section{User stories / behaviours}
\label{behaviours}

\begin{itemize}
\item As a user I want to.
\item As a user I want to.
\item As a user I want to.
\end{itemize}

\section{Comparing the technologies}
Figure \ref{tab:compare} on the next page compares the technologies discussed so far in the context of the behaviours in section \ref{behaviours}. \\
Looking at these row by row it can be seen that:\\\
\begin{itemize}
\item Bitcoin is too slow. Additionally the technical challenge of interacting with the network are considerable, and the fees are both high and rising.
\item Ethereum ERC721 non fungible tokens might be a possibility for supporting limited edition digital asset sales. A web interface could be built on cloud servers to handle the shop and back end commitments to the Ethereum network. While fully committed transactions would take a few minutes to process the shop itself could inform the users immediately then email out a certificate of ownership. It would be possible to trade these items on other platforms, and conceivable within the app. Prizes (digital assets) could be issued to users for competition activity. The system would be expensive to use, the fees might vary wildly, and the ecological costs might be enormous. Batching transactions could overcome this, but that would rely on substantial take-up of the service, and management overhead.
\item Building and issuing and Ethereum based token, in an ICO process or similar seems a disproportionate amount of investment and would be expensive to use. 
\item A custom private blockchain offers no advantages over a database. It is a difficult thing to implement, but would have better performance and use cost to the users.
\item BTC pay server is an interesting open source project which allows companies to host a simple lightning / BTC web shop. It's not clear what the use case is here, but it is well engineered and scalable and should be born in mind when thinking about the other problems and systems.
\item Lightning network has open source connectivity into Unity, works well with QR codes, is fast, and has low transaction latency. The only issue is that it offers nothing but marketing hype over and above more traditional digital point of sale layers, and is more difficult to connect to money rails in the UK.
\item Coloured coins would allow all of the desirable functionality, but it would be a substantial development challenge, using old open source code, and have high use fees. 
\item RGB MyCitadel stack seems highly appropriate and is open source but is in development and will not be available until later in the year.
\end{itemize}

%------------------------------------------------
\section{Risks}


