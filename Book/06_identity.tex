This is where an external contributor can add information about DID and SSI
\section{Some basics about distributed identity}
\subsection{DID principles}
persistent identifier (permanent)
Resolvable
cryptographically verifiable
decentralised
\subsection{primatives}
lifetime
portable
does not depend on third party
cannot be taken away
\subsection{What's in a DID document?}
it's to bootstrap the services
one or more public keys
One or more services which can be used
timestamps, sigs, proofs, delegations, authorisations
minimum amount of information
DID controller = DID subject
\subsection{What are the challenges?}
    fragmentations
    scaling to a world where the user is presented with thousands of signoffs
    how to visualise and abstract and glom this stuff
    How to care about the right level of information
\subsection{keystone uses}
\begin{itemize}
\item health
\item qualifications and certifications
\item finance
\item contacts
\item calendar
\end{itemize}  
\section{Technologies}
\subsection{Slashtags}
Slashtags is a new distributed identity open method being developed by Bitfinex and Tether under the Synonym suite. It uses bitcoin keys for authentication, but communicates a schema through a metadata exchange.
\subsection{LNURL-Auth}
%\lipsum[50]
\subsection{Sovrin}
Sovrin foundation are established global players in distributed identity and we wish to apply to join their federation of credential issuers.
\subsection{Keri}
%\lipsum[50]
\subsection{Atala Prism}
%\lipsum[50]
\subsection{Microsoft ION}
While working at Microsoft on ION Daniel Buchner (now working at Square) or Henry Tsai \href{https://github.com/decentralized-identity/ion/blob/master/docs/Q-and-A.md}{said the following}, which is worth quoting verbatim:\\
``While ledger-based consensus systems, on the surface, would seem to provide the same general features as one another, there are a few key differences that make some more suitable for critical applications, like the decentralized identifiers of human beings. Some of these considerations and features are:
\begin{itemize}
\item The system must be open and permissionless, not a cabal of authorities who can exclude and remove participants.
\item The system must be well-tested, and proven secure against attack over a long enough duration to be confident in.
\item The system must produce a singular, independently verifiable record that is as immutable as possible, so that reversing the record the system produces is infeasible.
\item The system must be widely deployed, with nodes that span the globe, to ensure the record is persisted.
\item The system must be self-incentivized, so that nodes continue to operate, process, and secure the record over time. The value from operation must come from the system directly, because outside incentive reliance is itself a vector for attack.
\item The cost to attack the system through any game theoretically available means must be high enough that it is infeasible to attempt, and even if an ultra-capitalized attacker did, it would require a weaponized mobilization of force and resources that would be obvious, with options for mitigation.\\

The outcome:

\item Number 1 eliminates private and permissioned ledgers
\item Number 2 eliminates just about all other ledgers and blockchains, simply because they are inadequately tested
\item For the metrics detailed in 3-6, Bitcoin is so far beyond all other options, it isn't even close - Bitcoin is the most secure option by an absurdly large margin.''
\end{itemize}

On the surface then it might seem that the choice is Bitcoin again, and indeed that the open source Microsoft ION stack is a natural choice. 

\subsection{Atala Prism (ADA Ecosystem)}
\href{https://medium.com/coinmonks/part-2-using-nethereum-in-unity-5b09f2d8c718}{This Medium post} lists the pre-requisites required for interaction with Eth within Unity. At this stage this hasn't been expanded out. It's based around the .NET `nethereum' library which can be found on their \href{https://github.com/Nethereum/Nethereum}{github}.    
   
  

%------------------------------------------------
\section{Risks}


