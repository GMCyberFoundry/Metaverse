\section{Digital Land Metaverses}
One of the most intuitive ways to view a metaverse is as a virtual landscape. This is how metaverse was portrayed in the original Neal Stephenson use of the word. 
\section{Global enterprise perspective}
Microsoft have just bought Activision / Blizzard for around seventy billion dollars. This has been communicated by Microsoft executives as a ''Metaverse play'', leveraging their internal game item markets, and their massive multiplayer game worlds to build toward a closed metaverse experience like the one Meta is planning.
This builds on the success of early experiments like the Fornite based music concerts, which attracted millions of concurrent users to live events.

\section{NFT as metaverse narrative}
Within the NFT community it is normalised to refer to ownership of digital tokens as participation in a metaverse. 
This CNBC article highlights the confusion, as this major news outlet refers to \href{https://www.cnbc.com/2022/01/16/walmart-is-quietly-preparing-to-enter-the-metaverse.html}{Walmart prepares to offer NFTs}'' as an entry ``into the metaverse''.

\section{MMORG games and NFTs}
Traditional gamers have pushed back on the seemingly useful idea of integrating NTFs with traditional games. This may be in part because Ethereum mining has kept graphics card prices high for a decade.

\href{https://www.prnewswire.com/news-releases/hbar-foundation-and-ubisoft-partner-to-support-growth-of-gaming-on-hedera-network-301474971.html}{HBAR partnerships}
The \href{https://twitter.com/justinkan/status/1491270239967154178}{following text} is from Justin Kan, co-founder of twitch: \textit{``NFTs are a better business model for games. Many gamers seem to be raging hard against game studios selling NFTs. But NFTs are also better for players. Here’s why I think blockchain games will be the predominant business model in gaming in ten years. NFTs are a better business model for funding games . Example: recently I invested in a new web3 game SynCityHQ. They are building a mafia metaverse and raised \$3M in their initial NFT drop.\\ NFTs give studios access to a new capital market for raising capital from the crowd.NFTs can be a better ongoing model for games. Web3 games will open economies, and by building the games on open and programmable assets (tokens + NFTs) they will create far more economic value than they could from any one game. Imagine Fortnite, but other developers can build experiences on top of the V-Bucks and skins. Epic would get a royalty every time any transaction happens. As big as Fortnite is today, Open Fortnite could be much bigger, because it will be a true platform. NFTs are better for gamers Allowing gamers to have ownership of the assets they buy and earn in game allows them to participate in the potential growth of a game. It lets gamers preserve some economic value when they switch to playing something new. But what about the criticisms of NFTs?\\
Here are my thoughts on the common FUDs: "It’s just a money grab on the part of the studios!"\\
Game studios already switched over to the model of selling in-game items, cosmetics, etc to players long ago. But currently the digital stuff players are buying isn’t re-sellable. NFT ownership is strictly better for players. "The games aren’t real games." This reminds me of the criticism of free-to-play in 2008, when the games were Mafia Wars / FarmVille. We haven’t had time for great developers to create incredible experiences yet. Everyone investing in games knows there are great teams building. "Game NFTs aren’t really decentralized because they rely on models / assets inside centralized game clients."
Crypto is as much a movement as it is a technology. Putting items on a blockchain is what gives people trust that they have participatory ownership...which make people willing to buy in to the game. These assets are “backed” by blockchain.
The fact that these item collections are NFTs will make other people willing to build on top of them. "NFTs are bad for the environment." Solana and L2s solve this. NFT games are better for players and for game developers. Like the free-to-play revolution changed gaming, so will blockchain. The games of the future will be fully robust, with open and programmable economies.}''




\label{behaviours}
\begin{itemize}
\item As a user I want to select a digital asset I find in the AR/VR world and then be offered an option to purchase the asset so I can look at it in my own spaces.
\item As a user I want to click on a digital asset I find in the AR app and be given the opportunity to buy it as a rare digital representation so that only I and a few others are provably certified to own.
\item As a user I would like to transfer economic value to people and entities I meet in the metaverse such that it is agreed by all parties quickly that value has been provably transferred.
\item As a user I would like to access an online marketplace in the metaverse where I swap and trade digital assets with other users so that I continue to feel engaged.
\item As a user I would like to create content (inside or outside of the metaverse) so I can take it to metaverse and monetise it.
\item As a content creator or influencer I would like to engage with live audiences within the metaverse, and moneytise my opinions and knowledge in real-time. I would like to have a way to split this money with co-collaborators in real time.
\end{itemize}
\section{Crypto metaverses}

\href{https://naavik.co/business-breakdowns/axie-infinity/#axie-decon=}{Report on Axie Infinity}

\href{https://www.thesun.co.uk/tech/17348918/pavia-metaverse-cardano-crypto-game/}{Pavia Metaverse}
Probably the best example in the market at this time with connecting users with one another through blockchain is \href{https://lightnite.io/}{Satoshis Games `Litenite'}. Litenite is a `battle royale' game which allows users to earn Satoshis through the Lightning network.\par
Similarly, and potentially more significantly, \href{https://twitter.com/zebedeeio/status/1512128093653196809}{Zebedee} have brought Lightning based micropayments to \href{https://zebedee.io/infuse/}{Counter Strike}, which adds a financial layer directly to eSport, itself a multi billion pound global industry.\par
The Zebedee model is interesting in that they provide simple onboarding, and management solutions, for gaming and metaverse application developers. There is doubtless an opportunity to utilise their business model in the proposed stacks in the paper, but it would be at odds with the free and open source product methodology in this paper. Their CTO said the following in a \href{https://lightningjunkies.net/lightning-address-making-lightning-user-friendly-lnj052/}{recent podcast}:\par
``We all had very similar ideas around being able to put Lightning network capabilities into games. You're essentially putting real value inside of the game. So whether it's a point inside of the game or as a real end game economy, as a game developer, you don't have to worry about the mechanics around it.\par
You can use a real life currency: the money that exists in the world and that value carries regardless of which game you're in, regardless if you're in the real world or inside of a virtual world, like a game. We just think that vision is just sort of opening. Now there's so much more, if you extrapolate it into the Meatverse we would love for Zebedee to be a big platform provider and enabler for a lot of these.''\par
\href{https://spellsofgenesis.com/}{Spells of Genesis} is a long running card RPG trading game on mobiles which allows ``ownership'' of items and cards through non-fungible token ecosystems.\par
There are also hundreds of casinos which operate within and even on blockchain networks. These feel out of scope as they are a different and somewhat regulated offering.
