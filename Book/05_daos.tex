A DAO is an organistion which is built in distributed code on a blockchain smart contract system. Token holders have voting rights proportional to their holding. The first decentalised autonomous organisation was simply called ``The DAO'' and was launched on the Ethereum network in 2016 after raising around \$100M. \href{https://www.gemini.com/cryptopedia/the-dao-hack-makerdao#section-what-is-a-dao}{It quickly succumbed to a hack and the money was drained}. This event was an important moment in the development of Ethereum and resulted in a code fork which preserves two separate versions of the network to this day, though one is falling into obsolescence. \\
In practice DAOs have very few committed `stakeholders' and the same names seem to crop up across multiple projects. Some crucial community decisions within large projects only poll a couple of dozen eligible participants. Its might be that the experiment of distributed governance is failing at this stage. \par
Perhaps more interesting is the use of the DAO concept to crowd fund global projects, currently especially for the acquisition of important art or cultural items. DAOs are also emerging as a way to fund promising technology projects, though this is reminiscent of the 2017 ICO craze which ended badly and is likely to fall foul of regulations.\par
Within the NFT and digital art space  PleaserDAO has quickly established a strong following.
``PleasrDAO is a collective of DeFi leaders, early NFT collectors and digital artists who have built a formidable yet benevolent reputation for acquiring culturally significant pieces with a charitable twist.\par
Opensea wrangle between IPO and governance token.\par
ConstitutionDAO, Once upon a time in Shaolin etc 
%https://harpers.org/archive/2015/01/come-with-us-if-you-want-to-live/

\section{Bisq DAO}
One of the better designed DAOs is \href{https://bisq.network/dao/}{Bisq DAO}. It's slightly different design trys to address the issue of overly rigid software intersecting with more intangible and fluid human governance needs. From their website:\par
\textit{``Revenue distribution and decision-making cannot be decentralized with traditional organization structures—they require legal entities, jurisdictions, bank accounts, and more—all of which are central points of failure.
The Bisq DAO replaces such legacy infrastructure with cryptographic infrastructure to handle project decision-making and revenue distribution without such central points of failure.''}

\section{Risks}