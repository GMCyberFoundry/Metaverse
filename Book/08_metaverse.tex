
\section{History and market need}

The word metaverse was coined by the author Neal Stephenson in his 1992 novel Snowcrash. It started popping up soon after in \href{https://www.newscientist.com/article/mg14819994-000-how-to-build-a-metaverse/}{news articles} and research papers \cite{mclellan1993avatars}.

\href{https://techtelegraph.co.uk/remembering-vrml-the-metaverse-of-1995/}{Remembering VRML: The Metaverse of 1995}\\


The concept of the Metaverse is extremely plastic at this time (Figure \ref{fig:muskWeb3}). This section will attempt to frame the context, and explain the increasingly polarised options looking forward.\\

\begin{figure}
  \centering
    \includegraphics[width=\linewidth]{muskWeb3}
  \caption{Elon Musk agrees with this on Twitter}
  \label{fig:muskWeb3}
\end{figure}

Both Second Life and more recently the game Fortnite can be seen as precursors. Second Life should rightly be viewed as the first serious attempt at a metaverse, and was being described as such by users as early as \href{https://nwn.blogs.com/nwn/2003/06/the_early_creat.html}{2002/2003}. It broke through into academic research and several Universities bought `digital land' and started talking about using the platform for education \cite{sermon2008they, emp2006putting, kirriemuir2008spring, sant2009performance}. Businesses began to develop and showcase virtual products. Interest in the platform \href{https://trends.google.com/trends/explore?date=all&q=second\%20life}{waned by 2010}, although the platform is still operational and under development with \href{https://danielvoyager.wordpress.com/category/second-life-stats/}{around 40k} median concurrent users. Surprisingly this is similar to the 2007 peak use, but this is in the context of other platforms now boasting considerably faster growth (more later).\\
Epic games Tim Sweeney \href{https://venturebeat.com/2017/05/15/epic-games-tim-sweeney-fears-the-metaverse-will-be-a-proprietary-technology/}{attaching the word metaverse} to social events within fortnite in 2017.\\
``You’re seeing the beginning components of the Metaverse coming together now..'' - Tim Sweeney\\
Clearly he ignored the extensive work of Second Life here, but fortnite demonstrated a new level of social engagement boasting millions of concurrent users in a single space for major events in the game. Most interestingly events outside of the game logic emerged, with concerts drawing over \href{https://www.epicgames.com/fortnite/en-US/news/astronomical}{10M users on occasion}. This has kickstarted a new round of academic interest in the phenomenon \cite{marlatt2020capitalizing}. A new era of microtransactions for in game assets has begun, thought this is constrained to the walled economic garden within Epic's servers.

\section{Post `Meta' metaverse}
The current media around ``metaverse'' has been seeded by Mark Zuckerberg's rebranding of his Facebook company to `Meta', and his planned investment in the technology. The second order hype is likely a speculative play by major companies on the future of the internet. There has been a reactive pushback against this by the wider tech community who are concerns about monetisation of biometrics. \href{https://www.coindesk.com/layer2/2022/01/19/meta-leans-in-to-tracking-your-emotions-in-the-metaverse/}{Observers do not trust} these `Web2' players with such a potentially powerful social medium. It is very plausible that this is all just a marketing play that goes nowhere and fizzles out. It is by no means clear that people want to spend time socialising globally in virtual and mixed reality. These major companies are  making an asymmetric bet that if there is a move into virtual worlds, then they need to be stakeholders in the gatekeeping capabilities of those worlds.\\ 
Meta,Disney plus, Sportswear manufacturers\\

\href{https://medium.com/kabuni/fiction-vs-non-fiction-98aa0098f3b0}{Can enough be done to prevent abuse?}

It seems like there are four major interpretations of the word.\\

Facebook have recently rebranding their parent company as `Meta' and they are aggressively promoting ``The Metaverse'' as a shared social VR space, chiefly of their design. In Stephenson's `Snow Crash' the Hero Protagonist (drolly called Hiro Protagonist) spends much of the novel in a dystopian virtual environment called the metaverse. It is unclear if Facebook is deliberately embracing the irony of aping such a dystopian image, but certainly their known predisposition for corporate surveillance, alongside their attempt at a global digital money is ringing alarm bells.\\
The Grayscale investment trust \href{https://grayscale.com/wp-content/uploads/2021/11/Grayscale_Metaverse_Report_Nov2021.pdf}{published a report} which views Metaverse as a potential trillion dollar global industry. Such industry reports are given to hyperbole, but it seems the technology is becoming the focus of technology investment narratives.
\subsection{Mixed reality as a metaverse}
\href{https://docs.microsoft.com/en-us/windows/mixed-reality/design/spatial-anchors}{Spatial anchors} allow digital objects to be overlaid persistently in the real world. With a global `shared truth' of such objects a different kind of metaverse can arise.
One such example is the forthcoming \href{https://avvyland.com/}{AVVYLAND}.

Peleton as a metaverse?

\section{Digital Land}
\lipsum[50]
\subsection{Secondlife}
\lipsum[50]
\subsection{The new stuff}
\lipsum[50]
\section{Global enterprise perspective}
Meta(Facebook)

Nvidia Omniverse is \href{https://www.nvidia.com/en-us/omniverse/creators/}{free for creators}, Unity etc \\

Microsoft have just bought Activision / Blizzard for around seventy billion dollars. This is have been communicated by Microsoft executives as a ''Metaverse play'', leveraging their internal game item markets, and their massive multiplayer game worlds to build toward a closed metaverse experience like the one Meta is planning.
This builds on the success of early experiments like the Fornite based music concerts, which attracted millions of concurrent users to live events.

\section{NFT as metaverse narrative}
Within the NFT community it is normalised to refer to ownership of digital tokens as participation in a metaverse. 
This CNBC article highlights the confusion, as this major news outlet refers to \href{https://www.cnbc.com/2022/01/16/walmart-is-quietly-preparing-to-enter-the-metaverse.html}{Walmart prepares to offer NFTs}'' as an entry ``into the metaverse''.
\lipsum[50]
\section{MMORG games and NFTs}
Traditional gamers have pushed back on the seemingly useful idea of integrating NTFs with traditional games. This may be in part because Ethereum mining has kept graphics card prices high for a decade.

\href{https://www.prnewswire.com/news-releases/hbar-foundation-and-ubisoft-partner-to-support-growth-of-gaming-on-hedera-network-301474971.html}{HBAR partnerships}
\label{behaviours}
\begin{itemize}
\item As a user I want to select a digital asset I find in the AR/VR world and then be offered an option to purchase the asset so I can look at it in my own spaces.
\item As a user I want to click on a digital asset I find in the AR app and be given the opportunity to buy it as a rare digital representation so that only I and a few others are provably certified to own.
\item As a user I would like to transfer economic value to people and entities I meet in the metaverse such that it is agreed by all parties quickly that value has been provably transferred.
\item As a user I would like to access an online marketplace in the metaverse where I swap and trade digital assets with other users so that I continue to feel engaged.
\item As a user I would like to create content (inside or outside of the metaverse) so I can take it to metaverse and monetise it.
\item As a content creator or influencer I would like to engage with live audiences within the metaverse, and moneytise my opinions and knowledge in real-time. I would like to have a way to split this money with co-collaborators in real time.
\end{itemize}
\section{Crypto metaverses}

\href{https://naavik.co/business-breakdowns/axie-infinity/#axie-decon=}{Report on Axie Infinity}

\href{https://www.thesun.co.uk/tech/17348918/pavia-metaverse-cardano-crypto-game/}{Pavia Metaverse}
Probably the best example in the market at this time with connecting users with one another through blockchain is \href{https://lightnite.io/}{Satoshis Games `Litenite'}. Litenite is a `battle royale' game which allows users to earn Satoshis through the Lightning network.\\
Similarly, and potentially more significantly, Zebedee have brought Lightning based micropayments to \href{https://zebedee.io/infuse/}{Counter Strike}, which adds a financial layer directly to eSport, itself a multi billion pound global industry.\\
\href{https://spellsofgenesis.com/}{Spells of Genesis} is a long running card RPG trading game on mobiles which allows ``ownership'' of items and cards through non-fungible token ecosystems.\\
There are also hundreds of casinos which operate within and even on blockchain networks. These feel out of scope as they are a different and somewhat regulated offering.

\section{Social VR software options}
In considering the needs of business to business and business to client social VR is it useful to compare software platforms:
\subsection{Second Life}
Notable because it's the original and has a decently mature marketplace.
\lipsum[50]
\subsection{Spatial}
\begin{itemize}
\item Very compelling. Wins at wow.
\item Great avatars, user generated
\item AR first design
\item Limited scenes
\item Smaller groups (12?)
\item Limited headset support
\item Intuitive meeting support tools
\item No back end integration
\end{itemize}
\subsection{MeetinVR}
\begin{itemize}
\item Good enough graphics, pretty mature system
\item OK indicative avatars, user selected
\item VR first design
\item Limited scenes
\item Smaller groups (12?)
\item Quest and PC
\item Writing and gestures supported
\item Some basic enterprise tools integration
\item Bring in 3D objects
\item Need to apply for a license?
\end{itemize}
\subsection{Glue}
\begin{itemize}
\item Better enterprise security integration
\item Larger environments, potential for breakouts in the same space. Workshop capable
\item 3D object support, screen sharing, some collaborative tools
\item Apply for a license
\item Fairly basic graphics
\item Basic avatars
\item Quest and PC
\item Writing and gestures supported
\item Mac support
\end{itemize}
\subsection{Mozilla Hubs}
\begin{itemize}
\item Open source, bigger scale, more complex
\item Choose avatars, or import your own
\item Environments are provided, or can be designed
\item Useful for larger conferences with hundreds or thousands of members but is commensurately more complex
\item Quest and PC
\item Larger scenes within scenes
\end{itemize}
\subsection{FramesVR}
\begin{itemize}
\item Really simple to join
\item Basic avatars
\item Bit buggy
\item 3D object support, screen sharing, some collaborative tools
\item Quest and PC
\item Larger scenes within scenes
\item Runs in the browser
\end{itemize}
\subsection{AltSpace}
\begin{itemize}
\item Microsoft social meeting platform
\item Very good custom avatar design
\item Great world building editor in the engine
\item Doesn't really support business integration so it's a bit out of scope
\item Huge numbers (many thousands) possible so it's great for global events
\item Mac support
\end{itemize}
\subsection{Engage}
\begin{itemize}
\item Great polished graphics
\item Fully customisable avatars
\item Limited scenes
\item Presentation to groups for education and learning
\item PC first, quest is side loadable but that's a technical issue
\item BigScreen VR
\item Seated in observation points in a defined shared theatre
\item Screen sharing virtual communal screen watching, aimed at gamers, film watching
\item up to 12 user
\end{itemize}
\subsection{VRChat}
\lipsum[50]
\subsection{NEOSVR}
\href{https://neos.com/}{Notable becasue} it's trying to integrate crypto marketplaces
\subsection{Facebook Horizon Workrooms}
\lipsum[50]
\subsection{Vircadia}
We have chosen Vircadia as our development platform for this investigation is it's a community supported free and open source project with some support for economic interaction.
\subsection{WebXR}
\href{https://hackmd.io/@XR/monetization}{Monetisation of WebXR}
\lipsum[50]
\subsection{Integration with web and game engines} (other integrations?) 
\section{User stories}
\lipsum[50]
\section{Recommendations for value transfer}
There are many claims about what blochchain technologies can enable. In the tens of thousands of attempts at utility over the last decade there are surprisingly few chains able to claim any value at all, and those that can often have significant problems in other areas. It is not the intention of this analysis to poke at problems. The primary use case within the context of a shared social space (metavserse) is low friction transmission of value. Remember that interlocutors and entities might have globally different physical locations outside of the real world. What is needed is instant ``settlement'' of digital bearer instruments within the context of the digital environement. This is fortunately what Bitcoin was designed to do. Additionally it is highly possible that exchange of digital goods (art and objects and provable services) will be required. This is also in scope within this document.
\section{Bitcoin market gap}
Nobody is currently deploying the discussed technologies purely on Bitcoin because it's slower evolution is still catching up. This is a market gap and the following section shows how this might be done.
\subsection{Money}
\lipsum[50]
\subsection{Identity proofs}
\lipsum[50]
\subsection{Digital object tracking}
\lipsum[50]
\subsection{Object transfer and trading}
\lipsum[50]
\section{Risks}