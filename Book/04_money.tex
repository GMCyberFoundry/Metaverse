It is necessary here to briefly examine what money actually is. In the previous section Bitcoin can be viewed in a couple of different lights. As a self custody digital bearer asset it can be viewed as `property', like gold. Indeed this has long been one of the assertions of the community and it finds favour in law. `Money' though is a far more \href{https://www.bankofengland.co.uk/knowledgebank/what-is-money}{slippery concept} to grasp. It seems very likely that Bitcoin is evolving as a money, and it's important to define that, but there are many other kinds of money within the online world which can potentially transfer value within virtual social spaces.
\section{Defining money}
It is \href{https://www.lynalden.com/what-is-money/}{hard to find} a universally accepted definition of what money is. The best approach is to look at the properties of a thing which is asserted to be a money. In his book `A history of money', Glyn Davies identifies ``cognisability, utility,  portability, divisibility, indestructibility, stability of value, and homogeneity'' \cite{davies2010history}.\par
Stroukal examines Bitcoins' likely value as a money from an Austrian economics perspective and identifies ``portability, storability, divisibility, recognizability, homogeneity and scarcity'' \cite{stroukal2018can}.\par
A helpfully brief and useful \href{http://money.visualcapitalist.com/infographic-the-properties-of-money/}{web page by Desjardins from 2015} describes some properties and explains them in layman's terms below:
\begin{itemize}
\item Divisible: Can be divided into smaller units of value.
\item Fungible: One unit is viewed as interchangeable with another.
\item Portable: Individuals can carry money with them and transfer it to others.
\item Durable: An item must be able to withstand being used repeatedly.
\item Acceptable: Everyone must be able to use the money for transactions.
\item Uniform: All versions of the same denomination must have the same purchasing power.
\item Limited in Supply: The supply of money in circulation ensures values remain relatively constant.
\end{itemize}
\subsection{Global categories of capital}
The legacy moniker ``third world'' came from a division of the world along economic lines \cite{tomlinson2003third}. At the time this was the petrodollar / neo-institutional hegemony \cite{caballero2008financial, spiro2019hidden}, vs the economic superpower of the soviet block, and then `the rest'; unaligned economic powers.\par
This old framework has fallen away with the associated terminology, but it's useful to look at what money `is' from a global viewpoint, because all money is effectively trust in the liability held by some defined counter party.\par
Right now the dollar system is still predominant, but it seems likely that there are new axes forming, especially around the \href{https://www.wsj.com/articles/saudi-arabia-considers-accepting-yuan-instead-of-dollars-for-chinese-oil-sales-11647351541}{Chinese Yuan}. It's clear that central banks have been aware of this potential transition away from a global dollar / energy system. Policy makers have been looking back to the great economic John Maynard Keynes' ideas for a neutral basket of assets as a global synthetic hedgemonic currency \cite{carney2019growing, piffaretti2009reshaping} which would almost certainly consist partly of gold \cite{stoeferle2018gold}.\par
Use of the dollar system has recently been shown more and more to be contingent on adherence to US defined political principles. This is evidenced most starkly by the seizure of Russian central bank \href{https://twitter.com/RussianEmbassy/status/1504530573527760909}{foreign reserves}, a new and untried projection of monetary power.\par  
The Chinese Yuan/Renminbi is potentially stepping in where the petrodollar is now waning \cite{mathews2018china}. The effects of this expansion of economic influence by China, through a potential petro-Yuan, and the belt and road initiative \cite{huang2016understanding}, are not yet felt, but the lines are fairly clearly defined. The Euro system is potentially more stable, and less `weaponised' but comes with it's own restrictions for use, especially through the International Monetary Fund (IMF). It is notable for instance that the IMF have included a clause in their negotiations with Argentina to `discourage' the use of crypto based momey. This is likely a response to the adoption of Bitcoin by El Salvador, something which the IMF is very uncomfortable. \par
The new `third world' who are excluded from the Dollar and/or Yuan poles of the global economy might drift toward the `basket of assets' discussed by Keynes and Carney above. As mentioned this will certainly have a component of gold, and likely other commodity assets such as rare metals. For our purposes here it's also possible that there would be a small `hedge' allocation of Bitcoin or even a global axis of `unaligned' nations using the asset \cite{hendrickson2021value}. This is evidenced in the early nation state adoption seen and described to date, and the game theory incentive explained by Fidelity in the introduction. It's too early to tell if this `unaligned money' could constitute a global economic pole, but it's interesting that some commentators are now even discussing this.

\section{International money transfer networks}
Transferring money from one financial jurisdiction to another is itself a global marketplace which has accreted over the entire course of human history. It's far less useful here to discuss the mythos of salt and seashells as a mechanisms of international remittance and taxation \cite{gainsford2017salt, goldberg2005famous}. Suffice it to say that there are dozens, if not hundreds, of cross border payment companies who make their business from taking a percentage cut of an international money transfer. There are also hundreds not thousands of banks who offer this service as part of their core business portfolio. This section looks at some of the major players, and their mechanism, to contextualise the more recent shifts brought about by technology.
\subsection{Swift, ISO 20022, and correspondence banking}
Society for Worldwide Interbank Financial Communiactions (SWIFT) was initially formed in 1973 between 239 banks across 15 countries. They needed a way to improve handling of cross border payments. It is now the global \href{https://www.swift.com/standards}{standard} for financial message exchange in over 200 countries, and has recently found itself under a fresh spotlight, during the invasion of Ukraine. The system handles around 40 million short, secure, code transmissions a day, which represent crucial data about a transaction and the parties involved. It is used by both banks and major financial institutions to speed up settlement between themselves, on behalf of the clients and customers. It replaced the Telex (wire transfer) system. The new proposed and incoming standard to replace SWIFT is \href{https://www.swift.com/standards/iso-20022}{ISO20022} which is a complex and data rich arrangement. To be clear the SWIFT consortium are promoting this new standard to their 11,000 plus global user base, and there is significant investment and hype from major financial players, but it seems unclear what the actual take-up will or even should be. A group of `crytocurrencies' are heavily involved in the ISO20022 standard, and there's been experimentation with private permissioned distributed ledger technologies. It's actually somewhat unclear what value they bring, and possible that the relationship of these public ledgers to international bank to bank messaging is a marketing distraction. Note that SWIFT, ISO20022, and the associated tokens within crypto are all themselves products which have a business model. They are all intermediaries which will demand a mediating fee somewhere. All of this proposed functionality could be replaced by central bank digital currencies, which will be discussed later in the section.
\subsection{VISA etc}
VISA have announced a ``\href{https://investor.visa.com/news/news-details/2021/Visa-Introduces-Crypto-Advisory-Services-to-Help-Partners-Navigate-a-New-Era-of-Money-Movement/default.aspx}{crypto business to business support unit}''. 
\subsection{Money transfer operators}

\href{https://www.toptal.com/finance/market-research-analysts/international-money-transfer}{International Money Transfer Operators analysis}

western union etc, moneygram, transferwise,
\subsection{Digital disruptive fintech}
It seems that the neobank providers of digital banking apps are likely to converge with native digital asset ``wallets''. This is also the thesis advanced by the Ark intestments Big Ideas paper.\par
Strike is possibly the most interesting product in the international fintech arena. It is a `global' money transmitter which uses bank connections in local currencies, but a private version of the Lightning network with settlement on the Bitcoin main chain. In practice users connect the app to their bank and can send money to the bank connected Strike app of another user instantly, and without a fee. This is a far better product than those previously available. In principle it's open API allows many more applications to be integrated into the Strike back end. Twitter already uses this for international tipping (and remittance). It seems that this is a perfect contender for supporting transactions in open metaverse applications, and that may be true, but Strike is currently only available in three countries (USA, El Salvador, Argentina).\par
%Paypal, xoom, Strike, servicing smaller payments, cashapp, venmo, revulot, 
\subsection{Stablecoins}
Stablecoins are `crypto like' instruments which are `pegged' at a 1:1 ratio with nationally issued Fiat currencies. In fact they correspond to units of privately issued  debt underwritten by a variety of different assets. This is (depending on the issuing company's model) a far more risky unit of money than the nominal currency that they represent, but they offer significant utility. They allow the user to self custody the cryptographic bearer instrument representing the money themselves, as with blockchain. This may afford the user less friction in that they can transmit the instrument through the newer financial rails which are emerging. The caveat here is that such `units' of money can be frozen by the issuer, and they are subject to the third party risk of the issuer defaulting on the underlying instrument, instantly wiping out the value.\par
It's worth taking a look at these tokens individually, to get a feel for the trade-offs, and figure out how they might be useful for us in our proposed metaverse applications. It's important to know that these tokenised dollars and/or other currencies are issued on top of the public blockchains we have been detailing throughout. Which tokens are on what blockchains is constantly evolving, so it's not really worth enumerating specifics. In a metaverse application it would be necessary to manage both the underlying public blockchain and the stablecoin issued on top of it, making the interaction with the global financial system perversely more not less complex. In the following list of a few of the major coins, the first hyperlink is the whitepaper if it's available.
\begin{itemize}
\item \href{https://tether.to/en/whitepaper/}{Tether} is the largest of the stablecoins, with some \$80B in circulation. This has been a meteoric rise, attracting the ire and scrutiny of \href{https://www.cftc.gov/PressRoom/PressReleases/8450-21}{regulators} and \href{https://www.bloomberg.com/news/features/2021-10-07/crypto-mystery-where-s-the-69-billion-backing-the-stablecoin-tether}{investigators}. There is considerable doubt that Tether has sufficient assets backing their synthetic dollars, but the market seems not to mind. It's an important technology for this metaverse conversation because of intersections with Bitcoin through the Lightning network. Tether might actually provide everything needed. It's only as safe as the trust invested in the central issuer though.
\item \href{https://f.hubspotusercontent30.net/hubfs/9304636/PDF/centre-whitepaper.pdf}{USDC} is a dollar backed coin issued by a consortium of major players in the space, most notably Cicle, and Coinbase. It's has a better transparency record than tether but is still not backed 1:1 by actual dollars in reserve. It may or may not be a fractional reserve asset. It's  well positioned to take advantage of regulatory changes in the USA, and seems to be quietly lobbying to be the choice of a government endorsed digital dollar, at least a significant part of a central bank digital currency initiative. It's too early to tell how this will work out.
\item Binance USD is the dollar equivalent token from global crypto exchange behemoth Binance. It's released in partnership with Paxos, who have a strong record for compliance, and transparency. Paxos also offer USDP. Both these stablecoins claim to be 100\% backed by dollars, or US treasuries. They are regulated under the more restrictive New York state financial services and have a monthly \href{https://paxos.com/attestations/}{attestation report}.
\item \href{https://assets.website-files.com/611153e7af981472d8da199c/618b02d13e938ae1f8ad1e45_Terra_White_paper.pdf}{TerraUSD} (UST) is a newer and more experimental stablecoin, and one of a set of currency representations within the network. It works in concert with the LUNA token on the Terra blockchain in order to keep it's dollar stability. It's not backed in the same way as the other tokens, instead relying on an arbitrage mechanism using LUNA. This is too complex to attempt to unpack here. There is concern that this model of `algorithmic stable coin' is unstable \cite{clements2021built}. The developers of the Terra say they are addressing this by \href{https://etherscan.io/address/0xad41bd1cf3fd753017ef5c0da8df31a3074ea1ea}{adding enormous amounts} (billions of dollar) of Bitcoin to the reserve assets of the ecosystem. This is becoming a \href{https://murrayrudd.substack.com/p/luna-price-model-update-22-feb-2022?s=r}{fascinating case study} in a Bitcoin backed dollar denominated token, and potentially a playbook for other institutions if it works out.   
\item \href{https://makerdao.com/en/whitepaper#abstract}{MakerDAO Dai} is an Ethereum based stablecoin and one of the older offerings. It's been `governed' by a DAO since 2014. `Excess collateral', above the value of the dai-dollars to be minted, is voted upon before being committed to the systems' cryptographic `vaults' as a backing for the currency. These dai can then be used across the Ethereum network. Despite the problems with DAOs, and the problems with Ethereum, DAI is well liked by its community of users and has a health billion dollars of issuance.
\item \href{https://trueusd.com/pdf/TUSD_WhitePaper.pdf}{TrueUSD} claims to be fully backed by US dollars, held in escrow. It runs on the Ethereum blockchain. They have attestation reports \href{https://real-time-attest.trustexplorer.io/truecurrencies}{available on demand} and claim fully insured deposits. It's not quite that simple in that a portion of the backing is `cash equivalents'.
\item \href{https://www.gemini.com/static/dollar/gemini-dollar-whitepaper.pdf}{Gemini GUSD} claim reserves are ``held and maintained at State Street Bank and Trust Company and within a money market fund managed by Goldman Sachs Asset Management, invested only in U.S. Treasury obligations.'' which seems pretty clear.
\end{itemize}  .

Koning has looked into the different \href{http://jpkoning.blogspot.com/2021/08/stablecoin-regulatory-strategies.html}{regulatory approaches} used by various stablecoins.\par

\begin{itemize}
\item The highly regulated New York state financial framework (Paxos, Gemini)
\item Piggyback off of a (Nevada) state-chartered trust [TrueUSD, HUSD]
\item Get dozens of money transmitter licenses [USDC]
\item Stay offshore [Tether]
\end{itemize}

It's worth pointing out that Meta (Facebook at the time) had aspirations for a global stablecoin cryptocurrency called 
\href{https://bitcoin-red.com/facebook-libra-the-inside-story-of-how-the-companys-cryptocurrency-dream-died/}{Libre}, 
  
\href{https://www.usdfconsortium.com/}{USDF bank issued private dollar stablecoin}

\href{https://www.bloomberg.com/news/articles/2022-01-07/paypal-is-exploring-launc,h-of-own-stablecoin-in-crypto-push}{Paypal}

Whatsapp, Novi, USDP etc
Crypto dollarisaton (myanmar)
\href{https://twitter.com/Stacks/status/1409996245096148998}{USDC on Bitcoin?}


\section{Central bank digital currencies}
If 2021 was the year of the stablecoin then 2022 is likely to be the year of the central bank digital currency (CBDC). CBDCs would likely not exist without the \href{https://www.theguardian.com/world/2021/jul/09/currency-and-control-why-china-wants-to-undermine-bitcoin}{pressure exerted on central banks} by the concept of Bitcoin, and the stablecoins which emerged from the technology. \par
It now seems plausible that the world is moving toward a plurality of national and private currencies. This text from the \href{https://voxeu.org/article/benefits-central-bank-digital-currency}{thinktank VoxEU} highlights the pressure on central banks not to be `left behind':\par
\textit{``Given the rapid pace of innovations in payments technology and the proliferation of virtual currencies such as bitcoin and ethereum, it might not be prudent for central banks to be passive in their approach to CBDC. If the central bank does not produce any form of digital currency, there is a risk that it loses monetary control, with greater potential for severe economic downturns. With this in mind, central banks are moving expeditiously when they consider the adoption of CBDC.''}\par
CBDCs are wholly digital representations of national currencies, and as such are centralised database entries, endorsed and potentially issued by national governments. It is a rapidly evolving space, and many nations are now scrambling to \href{https://twitter.com/GobiernoMX/status/1476376240873517061}{catch up}.\par
The following text is taken from the March 2021 Biden government ``executive order'' on digital assets, and defines the current global legislative position well.\\
\textit{``Sec. 4.  Policy and Actions Related to United States Central Bank Digital Currencies.  (a)  The policy of my Administration on a United States CBDC is as follows:\\
(i) Sovereign money is at the core of a well-functioning financial system, macroeconomic stabilization policies, and economic growth.  My Administration places the highest urgency on research and development efforts into the potential design and deployment options of a United States CBDC.  These efforts should include assessments of possible benefits and risks for consumers, investors, and businesses; financial stability and systemic risk; payment systems; national security; the ability to exercise human rights; financial inclusion and equity; and the actions required to launch a United States CBDC if doing so is deemed to be in the national interest.\\
(ii)   My Administration sees merit in showcasing United States leadership and participation in international fora related to CBDCs and in multi‑country conversations and pilot projects involving CBDCs.  Any future dollar payment system should be designed in a way that is consistent with United States priorities (as outlined in section 4(a)(i) of this order) and democratic values, including privacy protections, and that ensures the global financial system has appropriate transparency, connectivity, and platform and architecture interoperability or transferability, as appropriate.\\
(iii)  A United States CBDC may have the potential to support efficient and low-cost transactions, particularly for cross‑border funds transfers and payments, and to foster greater access to the financial system, with fewer of the risks posed by private sector-administered digital assets.  A United States CBDC that is interoperable with CBDCs issued by other monetary authorities could facilitate faster and lower-cost cross-border payments and potentially boost economic growth, support the continued centrality of the United States within the international financial system, and help to protect the unique role that the dollar plays in global finance.  There are also, however, potential risks and downsides to consider.  We should prioritize timely assessments of potential benefits and risks under various designs to ensure that the United States remains a leader in the international financial system.''}\par

In traditional nation state currencies the central banks \href{https://www.bankofengland.co.uk/markets/bank-of-england-market-operations-guide}{control the amount} of currency in circulation by issuing debt to private banks, which is then loaned out to individuals. The debt is `destroyed' on the balance sheet to remove currency through the reverse mechanism. They also facilitate government debt \cite{filardo2012central}, and work (theoretically) outside of political control to adjust interest rates, in order to manage growth and flows of money. \par
Many things which cannot be done with traditional nation state money systems are possible with CBDCs, because they \href{https://voxeu.org/article/benefits-central-bank-digital-currency}{remove the middleman} of private banking between the end user and the policy makers. 
\begin{itemize}
\item Negative interest rates are possible, such that all of the money can lose purchasing power over time, and at a rate dictated by policy. This ``removal of the lower bound'' has been discussed by economists over the last couple of decades as interest rate mechanisms have waned in efficacy. It is not possible in the current system, and instead money must be added through \href{https://www.bankofengland.co.uk/monetary-policy/quantitative-easing}{quantitative easing}, which disproportionately benefits some though Cantillon effects \cite{cantillon1756essai, bordo1983some}.  
\item Ubiquitous basic income is possible in that money can be issued directly from government to all approved citizens, transferring spending power directly from the government to the people. This also implies efficiency savings for social support mechanisms.
\item Asset freezing and confiscation are trivial if CBDCs can replace paper cash money completely, as a bearer asset. Criminals and global `bad actors' could have their assets temporarily or permanently removed, centrally, by suspending the transferability of the digital tokens.
\item Targeted bailouts for vital institutions and industries are possible directly from central government policy makers. Currently private banks must be incentivised to make cheap loans available to sectors which require targeted assistance.
\item Financial surveillance of every user is possible. In this way a `panopticon of money' can be enacted, and spending rulesets can be applied. For instance, social support money might only be spendable on food, and child support only on goods and services to support childcare. This is a very dystopian set of ideas. Eswar Prasad says ``In authoritarian societies, central bank money in digital form could become an additional instrument of government control over citizens rather than just a convenient, safe, and stable medium of exchange.'' \cite{prasad2021future}
\item It's a virtually cost free medium of exchange, since there is no physical instrument which must be shipped, guarded, counted, assayed, and securely destroyed.
\item The counterfeiting risk is significantly reduced because of secure cryptographic underpinnings rather than paper or plastic anti counterfeiting technologies.
\item Global reach and control is instantly possible for the issuer. This is a big problem especially for a reserve currency such as the dollar. Two thirds of \$100 bills are \href{https://www.federalreserve.gov/pubs/ifdp/2012/1058/default.htm}{thought to} reside outside of the USA.
\item System level quantitative easing and credit subsidies are made far simpler and less wasteful when centrally dictated.
\item Transfer of liability and risk to the holder globally reduces the management costs for global deposits of a currency. 
\item It may be possible to automate the stability of a currency through continuous adjustment of the `peg' through algorithms or AI.
\end{itemize}

The UK has signalled that it is not interested in developing a CBDC at this time. It is viewed as a \href{https://committees.parliament.uk/publications/8443/documents/85604/default/}{solution in search of a problem}, with the Lords economic affairs committee saying:\par
\textit{``The introduction of a UK CBDC would have far-reaching consequences for households, businesses, and the monetary system for decades to come and may pose significant risks depending on how it is designed. These risks include state surveillance of people’s spending choices, financial instability as people convert bank deposits to CBDC during periods of economic stress, an increase in central bank power without sufficient scrutiny, and the creation of a centralised point of failure that would be a target for hostile nation state or criminal actors.''}\par
In the USA this text from Congressman Tom Emmer shows how complex and interesting this debate is becoming.\par
\textit{Today, I introduced a bill prohibiting the Fed from issuing a central bank digital currency directly to individuals. Here’s why it matters:\\
As other countries, like China, develop CBDCs that fundamentally omit the benefits and protections of cash, it is more important than ever to ensure the United States’ digital currency policy protects financial privacy, maintains the dollar’s dominance, and cultivates innovation.\\
CBDCs that fail to adhere to these three basic principles could enable an entity like the Federal Reserve to mobilize itself into a retail bank, collect personally identifiable information on users, and track their transactions indefinitely.\\
Not only does this CBDC model raise ``single point of failure'' issues, leaving Americans’ financial information vulnerable to attack, but it could be used as a surveillance tool that Americans should never be forced to tolerate from their own government.\\
Requiring users to open an account at the Fed to access a United States CBDC would put the Fed on an insidious path akin to China’s digital authoritarianism.\\
Any CBDC implemented by the Fed must be open, permissionless, and private. This means that any digital dollar must be accessible to all, transact on a blockchain that is transparent to all, and maintain the privacy elements of cash.\\
In order to maintain the dollar’s status as the world’s reserve currency in a digital age, it is important that the United States lead with a posture that prioritizes innovation and does not aim to compete with the private sector.\\
Simply put, we must prioritize blockchain technology with American characteristics, rather than mimic China’s digital authoritarianism out of fear.}
Most analysts now seem to think that there is little appetite to replace all of a given currency with a CBDC. It is far more likely that a blend of stablecoins, private bank issued digital currency (with a yield incentive) and some limited CBDC, alongside the new contender Bitcoin, will present a new landscape of user choice. Different models of trust, insurance, yields, acceptability, and potentially privacy, will emerge. \par
Clearly a global, stable, wholly digital bearer asset in a native currency would be the ideal integration for money in a metaverse application, but it is likely that a transition to such a technology would be complex and painful. It is certainly not ready for consideration now.

\href{https://www.federalreserve.gov/publications/files/money-and-payments-20220120.pdf}{US CBDC whitepaper}


\section{Bitcoin as a money}
Since this book seeks to examine transfer of value within a purely digital environment it is necessary to ask the question of whether Bitcoin is money. It is beyond argument that the Bitcoin network is a rugged message passing protocol which achieves a high degree of consensus about the entries on it's distributed database.\par Ascribing monetary value to those database entries is a social consensus problem, and this itself is a contested topic. \par
Jack Mallers, of Strike \href{https://www.youtube.com/watch?v=jb-45m9f76I}{presentation to the IMF} identified the following challenges which he claims are solved by the bitcoin monetary network.
\begin{itemize}
\item Speed
\item Limited transparency and dependability
\item High cost
\item Lack of interoperability
\item Limited Coverage
\item Limited accessibility
\end{itemize}
Mallers further identifies the attributes of the ideal global money. 
\begin{itemize}
\item Uncensorable
\item Unfreezable
\item Permissionless
\item Borderless
\item Liquid
\item Digital
\end{itemize}
The Bitcoin community believes that \href{https://svetski.medium.com/why-bitcoin-not-shitcoin-6cc826f4fa52}{Bitcoin is the ultimate money}, a \href{https://www.coindesk.com/business/2022/01/07/jpmorgan-sees-more-crypto-adoption-in-2022-debates-bitcoins-status-as-store-of-value/}{`store of value'}, chance to \href{https://www.forbes.com/sites/leeorshimron/2020/06/30/bitcoin-is-the-separation-of-money-and-state/?sh=49294a8356db}{separate money from state}, increase \href{https://www.washingtonpost.com/national/locked-out-of-traditional-financial-industry-more-people-of-color-are-turning-to-cryptocurrency/2021/12/01/a21df3fa-37fe-11ec-9bc4-86107e7b0ab1_story.html}{equality of opportunity} and \href{https://iai.tv/articles/the-rich-get-richer-the-poor-get-bitcoin-auid-1766}{ubiquity of access}, while others view it as \href{https://www.cnbc.com/2021/06/22/a-third-of-investors-think-bitcoin-is-rat-poison-jpmorgan-survey-says.html}{`rat poison'}, or a \href{https://jacobinmag.com/2022/01/cryptocurrency-scam-blockchain-bitcoin-economy-decentralization}{fraudulent Ponzi scheme}. A notable exclusion from the negative rhetoric is Fidelity, the global investment manager, who have always been positive and have \href{https://www.fidelitydigitalassets.com/articles/bitcoin-first?sf253214177=1}{recently said}: \\
\textit{``Bitcoin is best understood as a monetary good, and one of the primary investment theses for bitcoin is as the store of value asset in an increasingly digital world.''}\\
The following paraphrases Eric Yakes, author of \href{https://yakes.io/book/}{`The 7th Property'}. Again, this is an Austrian economics perspective, and like much economic theory the underlying premise \href{https://medium.datadriveninvestor.com/do-you-understand-the-austrian-vs-keynesian-economic-debate-2f4b152c6a6b}{is contested}\cite{maurel2012keynesian}.\par
\textit{``Paper became money because it was superior to gold in terms of divisibility and portability BUT it lacked scarcity. People reasoned that we could benefit from the greater divisibility/portability of paper money as long as it was redeemable in a form of money that was scarce. This is when money needed to be ``backed'' by something. \\
Since we changed money to paper money that wasn't scarce, it needed to be backed by something that was. Since the repeal of the gold standard, politicians have retarded the meaning of the word because our money is no longer backed by something scarce.\\
So, what is bitcoin backed by? Nothing.\\
Sound money, like gold, isn’t ``backed''.
Only money that lacks inherent monetary properties must be backed by another money that maintains those properties. The idea that our base layer money needs to be backed by something is thinking from the era of paper money. Bitcoin does not require backing, it has inherent monetary properties superior to any other form of money that has ever existed.''}\\

Perhaps more than any of these takes, it is worth considering the current public perception of the technology as a money and store of value. This \href{https://twitter.com/saquon/status/1480738426236375041}{twitter thread} from professional sportsman Saquon Barkley, to his half million followers on the platform, captures the mood. He is one of a handful of athletes now being \href{https://www.buybitcoinworldwide.com/athletes/}{paid directly} in Bitcoin.\par
\textit{``I want my career earnings to last generations. The average NFL career is 3 years and inflation is real. Saving and preserving money over time is hard, no matter who you are.
In today’s world: How do we save? This is why I believe in bitcoin. Almost all professional athletes make the majority of their career earnings in their 20s. With a lack of education, inaccessible tools, and inflation, a sad yet common reality is many enter bankruptcy later on. We can do better. We need to improve financial literacy. Bitcoin is a proven, safe, global, and open system that allows anyone to save money. It is the most accessible asset we’ve ever seen.''}\par

\href{https://andrewmbailey.com/}{Andrew M. Bailey} says \textit{``in an ideal world where governments honour the rights of citizens, they don't spy, they don't prohibit transactions, they manage a sound money supply, and they make sound decisions, the value of bitcoin is very low; we're just not in an ideal world''}\par

The 2022 ARK Big Ideas report again provides some useful market insight. The posit that demand for the money features of Bitcoin could drive the price of the capped supply tokens to around 1M pounds per Bitcoin as in Figure \ref{fig:BitcoinShareOfMoney}.

\begin{figure}
  \centering
    \includegraphics[width=\linewidth]{BitcoinShareOfMoney}
  \caption{Potential market exposure to Bitcoin as a money}
  \label{fig:BitcoinShareOfMoney}
\end{figure}

Fundamentally, Bitcoin isn't money (in the traditional sense) because it's not an IOU, which money certainly is. It's a bearer instrument, novel asset class, with money like properties, as identified above. Potentially the most important differentiating affordance is censorship resistance. There's really nothing else like it for that one feature. With that said Bitcoin is only a viable `money like thing' when viewed in the layers described in this book. The base chain layer is the ultimate store of secure value. Whatever layer 2 ultimately emerges is the transactional layer which could replace day to day cash money, while the hypothetical layer 3 might be useful for complex financial mechanisms and contracts operating automatically, and also provides the opportunity for using the security model of the chain to support other digital assets, including government currencies through stablecoins.
\section{Risks}
It can be seen that following the invasion of Ukraine by Russia, that sanctions of various kinds were applied to the Russian economy. One of these was the previously dicussed Swift international settlement network. Another whole catagory was the removal of support by private businesses domiciled outside of Russia and Ukraine, and pertinent here is that VISA, Mastercard, Paypal, and Western Union all removed support for their product rails. This means that while some cards and services still work, and will likely work again through Chinese proxies in the coming months, considerable disruption will be felt by Russian companies and individuals. This is not to say that this disruption is necessarily wrong, but it is clear now that all of these global financial transfer products and services are contingent on political factors. The same might be true of CBDC products if they grain traction globally. There is certainly no reason why all money within an physically delineated border could not be blocked or cancelled. This is not as true for Bitcoin at this time, though again, with enough political will it is technically plausible to incentivise miners with additional payments to exclude transactions from geolocated wallets. This would be mitigated by Tor, and in a global anonymous network it is very likely that a miner could be found at a higher price for inclusion in the next block. Finally a lack of fungibility and privacy by default trends towards blacklists and over time this could seriously compromise the use of the asset.
\subsection{Hyperbitcoinization}
Hyperbitcoinization is a term coined in 2014 by Daniel Krawisz \cite{krawisz2014hyperbitcoinization}. It is the hypothetical rise of Bitcoin to become the global reserve currency, and the demonetisation of all other store of value assets. This seems unlikely but is backed by a game theoretic analysis of both Bitcoin and current macro economics. Fulgur Ventures (a venture capital firm) provide a \href{https://medium.com/@fulgur.ventures/the-roads-to-hyperbitcoinization-part-1-27dc84d0e5e5}{blog post series} about the route this might take. It's beyond the scope of this paper to look at the implications of this possibility, but they are clearly significant if true. 

\section{Does DeFi matter to SMEs }
DeFi is decentralised finance, and might only exist because of partial regulatory capture of Bitcoin. If peer-to-peer Bitcoin secured yield and loans etc were allowed then it seems unlikely that the less secure and more convoluted DeFi products would have found a footing. DeFi  has been commonplace over the last few years. It enables trading of value, loans, and interest (yield) without onerous KYC. If Bitcoin's ethos is to develop at a slow and well checked rate, and Ethereum's ethos is to move fast and break things, then DeFi could best be described and throwing mud and hoping some sticks. It is characterised by rapid innovation, huge yields for early adopters, incredibly high risk, and a culture of speculation which leads to products being discarded and/or forked into something else in the pursuit of returns.\par 
Much of the space is now using arcane gamification of traditional financial tools, combined with memes, to promote what are essentially pyramid schemes. Scams are very commonplace. Loss of funds though code errors are perhaps even more prevalent.\par
The Bank for International Settlements have the stated aim of supporting central banks monetary and financial stability. Their \href{https://www.bis.org/publ/qtrpdf/r_qt2112b.pdf}{2021 report on DeFi} noted the following key problems.
\begin{itemize}
\item ..a ``decentralisation illusion'' in DeFi due to the inescapable need for centralised governance and the tendency of blockchain consensus mechanisms to concentrate power. DeFi`s inherent governance structures are the natural entry points for public policy.
\item DeFi’s vulnerabilities are severe because of high leverage, liquidity mismatches, built-in interconnectedness and the lack of shock-absorbing capacity.
\end{itemize}
These are two excellent and likely true points. In addition access to DeFi is `usually' through Web2.0 centralised portals (websites) which are just as vulnerable to legal takedown orders and any other centralised technology.\par
There are more recent DeFi on Bitcoin contenders, but these are vulnerable to the \href{https://bisq.community/t/trading-halted-until-v1-3-0-hotfix/9208}{same attacks} and problems in the main. \par 
There is likely no use for this technology for small and medium sized companies on the international stage. It is far more likely that reputation would be damaged. The `best' of the portfolio of DeFi offerings is perhaps high yield stablecoin accounts, where dollars equivalent tokens are locked up providing very high return rates of up to 20 percent. It's also possible to get loans (by extension business loans) out of such systems at relatively low risks. The best `distributed' example of this is probably \href{https://lend.hodlhodl.com/}{Lend, at HODLHODL}, which is a peer-to-peer loan marketplace. Many more custodial options exist for loans (CASA, BlockFi, Nexo, Ledn, Abra etc). These might not really fit the definition of DeFi at all, but they are potentially useful to companies who have Bitcoin on their balance sheet long term.

\section{Do DAOs matter to SMEs}
A DAO is an organistion which is built in distributed code on a blockchain smart contract system. Token holders have voting rights proportional to their holding. The first decentalised autonomous organisation was simply called ``The DAO'' and was launched on the Ethereum network in 2016 after raising around \$100M. \href{https://www.gemini.com/cryptopedia/the-dao-hack-makerdao#section-what-is-a-dao}{It quickly succumbed to a hack and the money was drained}. This event was an important moment in the development of Ethereum and resulted in a code fork which preserves two separate versions of the network to this day, though one is falling into obsolescence. \\
In practice DAOs have very few committed `stakeholders' and the same names seem to crop up across multiple projects. Some crucial community decisions within large projects only poll a couple of dozen eligible participants. Its might be that the experiment of distributed governance is failing at this stage. \\
Perhaps more interesting is the use of the DAO concept to crowd fund global projects, currently especially for the acquisition of important art or cultural items. DAOs are also emerging as a way to fund promising technology projects, though this is reminiscent of the 2017 ICO craze which ended badly and is likely to fall foul of regulations.\\
Within the NFT and digital art space  PleaserDAO has quickly established a strong following.
``PleasrDAO is a collective of DeFi leaders, early NFT collectors and digital artists who have built a formidable yet benevolent reputation for acquiring culturally significant pieces with a charitable twist.\\
Opensea wrangle between IPO and governance token.\\
ConstitutionDAO, Once upon a time in Shaolin etc 
%https://harpers.org/archive/2015/01/come-with-us-if-you-want-to-live/

\section{Bisq DAO}
One of the better designed DAOs is \href{https://bisq.network/dao/}{Bisq DAO}. It's slightly different design trys to address the issue of overly rigid software intersecting with more intangible and fluid human governance needs. From their website:\\
\textit{``Revenue distribution and decision-making cannot be decentralized with traditional organization structures—they require legal entities, jurisdictions, bank accounts, and more—all of which are central points of failure.\\
The Bisq DAO replaces such legacy infrastructure with cryptographic infrastructure to handle project decision-making and revenue distribution without such central points of failure.''}

\section{Risks}