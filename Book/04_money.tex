It is necessary here to briefly examine what money actually is. In the previous section Bitcoin can be viewed in a couple of different lights. As a self custody digital bearer asset it can be viewed as `property', like gold. Indeed this has long been one of the assertions of the community and it finds favour in law. `Money' though is a far more slippery concept to grasp. It seems very likely that Bitcoin is evolving as a money, and it's important to define that, but there are many other kinds of money within the online world which can potentially transfer value within virtual social spaces.
\section{Defining money}
It is hard to find a universally accepted definition of what money is. The best approach is to look at the properties of a thing which is asserted to be a money. In his book ‘A history of money’, Glyn Davies identifies ``cognisability, utility,  portability, divisibility, indestructibility, stability of value, and homogeneity'' \cite{davies2010history}.\\
Stroukal examines Bitcoins' likely value as a money from an Austrian economics perspective and identifies ``portability, storability, divisibility, recognizability, homogeneity and scarcity'' \cite{stroukal2018can}.\\
A helpfully brief and useful \href{http://money.visualcapitalist.com/infographic-the-properties-of-money/}{web page by Desjardins from 2015} describes some properties and explains them in layman's terms below:
\begin{itemize}
\item Divisible: Can be divided into smaller units of value.
\item Fungible: One unit is viewed as interchangeable with another.
\item Portable: Individuals can carry money with them and transfer it to others.
\item Durable: An item must be able to withstand being used repeatedly.
\item Acceptable: Everyone must be able to use the money for transactions.
\item Uniform: All versions of the same denomination must have the same purchasing power.
\item Limited in Supply: The supply of money in circulation ensures values remain relatively constant.
\end{itemize}
\section{International money transfer networks}
Transferring money from one financial jurisdiction to another is itself a global marketplace which has accreted over the entire course of human history. It's far less useful here to discuss the mythos of salt and seashells as a mechanisms of international remittance and taxation \cite{gainsford2017salt, goldberg2005famous}. Suffice it to say that there are dozens, if not hundreds, of cross border payment companies who make their business from taking a percentage cut of an international money transfer. There are also hundreds not thousands of banks who offer this service as part of their core business portfolio. This section looks at some of the major players, and their mechanism, to contextualise the more recent shifts brought about by technology.
\subsection{Swift, ISO 20022, and correspondence banking}
SWIFT is the global \href{https://www.swift.com/standards}{standard} for financial message exchange. It is a system of short codes which represent crucial data about transaction and the parties involved. It is used by banks and major financial institutions to speed up settlement between themselves, on behalf of the clients and customers. It replaced the Telex (wire transfer) system. The new proposed and incoming standard to replace SWIFT is \href{https://www.swift.com/standards/iso-20022}{ISO20022} which is a complex and data rich arrangement. To be clear the SWIFT consortium are promoting this new standard to their 10,000 plus global user base, and there is significant investment and hype from major financial players, but it seems unclear what the actual take-up will or even should be. A group of `crytocurrencies' are heavily involved in the ISO20022 standard, and there's been experimentation with private permissioned distributed ledger technologies. It's actually somewhat unclear what value they bring, and possible that the relationship of these public ledgers to international bank to bank messaging is a marketing distraction. Note that SWIFT, ISO20022, and the associated tokens within crypto are all themselves products which have a business model. They are all intermediaries which will demand a mediating fee somewhere. All of this proposed functionality could be replaced by central bank digital currencies, which will be discussed later in the section.

\lipsum[50]
\subsection{VISA etc}
VISA have announced a ``\href{https://investor.visa.com/news/news-details/2021/Visa-Introduces-Crypto-Advisory-Services-to-Help-Partners-Navigate-a-New-Era-of-Money-Movement/default.aspx}{crypto business to business support unit}''. 
\subsection{Money transfer operators}

\href{https://www.toptal.com/finance/market-research-analysts/international-money-transfer}{International Money Transfer Operators analysis}

western union etc, moneygram, transferwise,
\subsection{Digital disruptive fintech}
It seems that the neobank providers of digital banking apps are likely to converge with native digital asset ``wallets''. This is also the thesis advanced by the Ark intestments Big Ideas paper.\\
Strike is a possible the most interesting product in the international fintech arena. It is a `global' money transmitter which uses bank connections in local currencies, but a private version of the Lightning network with settlement on the Bitcoin main chain. In practice users connect the app to their bank and can send money to the bank connected Strike app of another user instantly, and without a fee. This is a far better product than those previously available. In principle it's open API allows many more application to be integrated into the Strike back end. Twitter already uses this for international tipping (and remittance). It seems that this is a perfect contender for supporting transactions in open metaverse applications, and that may be true, but Strike is currently only available in three countries (USA, El Salvador, Argentina).\\
Paypal, xoom, Strike, servicing smaller payments, cashapp, venmo, revulot, 
\lipsum[50]

\subsection{Stablecoins}
Stablecoins are `crypto like' instruments which are `pegged' at a 1:1 ratio with nationally issued Fiat currencies. In fact they correspond to units of privately issued  debt underwritten by a variety of different assets. This is (depending on the issuing company's model) a far more risky unit of money than the nominal currency that they represent, but they offer significant utility. They allow the user to self custody the cryptographic bearer instrument representing the money themselves, as with blockchain. This may afford the user less friction in that they can transmit the instrument through the newer financial rails which are emerging. The caveat here is that such `units' of money can be frozen by the issuer, and they are subject to the third party risk of the issuer defaulting on the underlying instrument.
\href{https://www.usdfconsortium.com/}{USDF bank issued private dollar stablecoin}

\href{https://www.bloomberg.com/news/articles/2022-01-07/paypal-is-exploring-launc,h-of-own-stablecoin-in-crypto-push}{Paypal}

Whatsapp, Novi, USDP etc
Crypto dollarisaton (myanmar)
\href{https://twitter.com/Stacks/status/1409996245096148998}{USDC on Bitcoin?}

\section{Central bank digital currencies}
If 2021 was the year of the stablecoin then 2022 is likely to be the year of the central bank digital currency (CDBC). These wholly digital representations of national currencies are oddities in that they are centralised database entries, endorsed and potentially issued by national governments. It is a rapidly evolving space and many nations are now scrambling to catch up. 
\href{https://twitter.com/GobiernoMX/status/1476376240873517061}{Central bank of Mexico}.\\
The UK has signalled that it is not interested in developing a CDBC at this time, with the Lords economic affairs committee saying:\\
\textit{``The introduction of a UK CBDC would have far-reaching consequences for households, businesses, and the monetary system for decades to come and may pose significant risks depending on how it is designed. These risks include state surveillance of people’s spending choices, financial instability as people convert bank deposits to CBDC during periods of economic stress, an increase in central bank power without sufficient scrutiny, and the creation of a centralised point of failure that would be a target for hostile nation state or criminal actors.''}\\
In the USA this text from Congressman Tom Emmer shows how complex and interesting this debate is becoming.\\
\textit{Today, I introduced a bill prohibiting the Fed from issuing a central bank digital currency directly to individuals. Here’s why it matters:\\
As other countries, like China, develop CBDCs that fundamentally omit the benefits and protections of cash, it is more important than ever to ensure the United States’ digital currency policy protects financial privacy, maintains the dollar’s dominance, and cultivates innovation.\\
CBDCs that fail to adhere to these three basic principles could enable an entity like the Federal Reserve to mobilize itself into a retail bank, collect personally identifiable information on users, and track their transactions indefinitely.\\
Not only does this CBDC model raise “single point of failure” issues, leaving Americans’ financial information vulnerable to attack, but it could be used as a surveillance tool that Americans should never be forced to tolerate from their own government.\\
Requiring users to open an account at the Fed to access a United States CBDC would put the Fed on an insidious path akin to China’s digital authoritarianism.\\
Any CBDC implemented by the Fed must be open, permissionless, and private. This means that any digital dollar must be accessible to all, transact on a blockchain that is transparent to all, and maintain the privacy elements of cash.\\
In order to maintain the dollar’s status as the world’s reserve currency in a digital age, it is important that the United States lead with a posture that prioritizes innovation and does not aim to compete with the private sector.\\
Simply put, we must prioritize blockchain technology with American characteristics, rather than mimic China’s digital authoritarianism out of fear.}

\href{https://www.federalreserve.gov/publications/files/money-and-payments-20220120.pdf}{US CDBC whitepaper}

\section{Bitcoin as a money}
Since this book seeks to examine transfer of value within a purely digital environment it is necessary to ask the question of whether Bitcoin is money. It is beyond argument that the Bitcoin network is a rugged message passing protocol which achieves a high degree of consensus about the entries on it's distributed database. Ascribing monetary value to those database entries is a social consensus problem, and this itself is a contested topic. \\
Jack Mallers, of Strike \href{https://www.youtube.com/watch?v=jb-45m9f76I}{presentation to the IMF} identified the following challenges which he claims are solved by the bitcoin monetary network.
\begin{itemize}
\item Speed
\item Limited transparency and dependability
\item High cost
\item Lack of interoperability
\item Limited Coverage
\item Limited accessibility
\end{itemize}
Mallers further identifies the attributes of the ideal global money. 
\begin{itemize}
\item Uncensorable
\item Unfreezable
\item Permissionless
\item Borderless
\item Liquid
\item Digital
\end{itemize}
The Bitcoin community believes that \href{https://svetski.medium.com/why-bitcoin-not-shitcoin-6cc826f4fa52}{Bitcoin is the ultimate money}, a \href{https://www.coindesk.com/business/2022/01/07/jpmorgan-sees-more-crypto-adoption-in-2022-debates-bitcoins-status-as-store-of-value/}{`store of value'}, chance to \href{https://www.forbes.com/sites/leeorshimron/2020/06/30/bitcoin-is-the-separation-of-money-and-state/?sh=49294a8356db}{separate money from state}, increase \href{https://www.washingtonpost.com/national/locked-out-of-traditional-financial-industry-more-people-of-color-are-turning-to-cryptocurrency/2021/12/01/a21df3fa-37fe-11ec-9bc4-86107e7b0ab1_story.html}{equality of opportunity} and \href{https://iai.tv/articles/the-rich-get-richer-the-poor-get-bitcoin-auid-1766}{ubiquity of access}, while others view it as \href{https://www.cnbc.com/2021/06/22/a-third-of-investors-think-bitcoin-is-rat-poison-jpmorgan-survey-says.html}{`rat poison'}, or a \href{https://jacobinmag.com/2022/01/cryptocurrency-scam-blockchain-bitcoin-economy-decentralization}{fraudulent Ponzi scheme}. A notable exclusion from the negative rhetoric is Fidelity, the global investment manager, who have always been positive and have \href{https://www.fidelitydigitalassets.com/articles/bitcoin-first?sf253214177=1}{recently said}:\\
\textit{``Bitcoin is best understood as a monetary good, and one of the primary investment theses for bitcoin is as the store of value asset in an increasingly digital world.''}\\
The following paraphrases Eric Yakes, author of \href{https://yakes.io/book/}{`The 7th Property'}. Again, this is an Austrian economics perspective, and like much economic theory the underlying premise \href{https://medium.datadriveninvestor.com/do-you-understand-the-austrian-vs-keynesian-economic-debate-2f4b152c6a6b}{is contested}\cite{maurel2012keynesian}.\\
\textit{``Paper became money because it was superior to gold in terms of divisibility and portability BUT it lacked scarcity. People reasoned that we could benefit from the greater divisibility/portability of paper money as long as it was redeemable in a form of money that was scarce. This is when money needed to be ``backed'' by something. \\
Since we changed money to paper money that wasn't scarce, it needed to be backed by something that was. Since the repeal of the gold standard, politicians have retarded the meaning of the word because our money is no longer backed by something scarce.\\ 
So, what is bitcoin backed by? Nothing.\\
Sound money, like gold, isn’t ``backed''.
Only money that lacks inherent monetary properties must be backed by another money that maintains those properties. The idea that our base layer money needs to be backed by something is thinking from the era of paper money. Bitcoin does not require backing, it has inherent monetary properties superior to any other form of money that has ever existed.''}\\
Perhaps more than any of these takes, it is worth considering the current public perception of the technology as a money and store of value. This \href{https://twitter.com/saquon/status/1480738426236375041}{twitter thread} from professional sportsman Saquon Barkley, to his half million followers on the platform, captures the mood. He is one of a handful of athletes now being \href{https://www.buybitcoinworldwide.com/athletes/}{paid directly} in Bitcoin.\\
\textit{``I want my career earnings to last generations. The average NFL career is 3 years and inflation is real. Saving and preserving money over time is hard, no matter who you are.
In today’s world: How do we save? This is why I believe in bitcoin. Almost all professional athletes make the majority of their career earnings in their 20s. With a lack of education, inaccessible tools, and inflation, a sad yet common reality is many enter bankruptcy later on. We can do better. We need to improve financial literacy. Bitcoin is a proven, safe, global, and open system that allows anyone to save money. It is the most accessible asset we’ve ever seen.''}\\

\href{https://andrewmbailey.com/}{Andrew M. Bailey} says \textit{``in an ideal world where governments honour the rights of citizens, they don't spy, they don't prohibit transactions, they manage a sound money supply, and they make sound decisions, the value of bitcoin is very low; we're just not in an ideal world''}\\

The 2022 ARK Big Ideas report again provides some useful market insight. The posit that demand for the money features of Bitcoin could drive the price of the capped supply tokens to around 1M pounds per Bitcoin as in Figure \ref{fig:BitcoinShareOfMoney}.

\begin{figure}
  \centering
    \includegraphics[width=\linewidth]{BitcoinShareOfMoney}
  \caption{Potential market exposure to Bitcoin as a money}
  \label{fig:BitcoinShareOfMoney}
\end{figure}

\section{Does DeFi matter to SMEs }
DeFi exists because of partial regulatory capture of Bitcoin. It has been commonplace over the last few yearsIt enables trading of value without onerous KYC. Expand. It is likely to be at risk.
\lipsum[50]

BIS report on DeFi
