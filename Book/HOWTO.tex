\hypertarget{lab---virtualisation-networking-bitcoin}{%
\section{Lab - virtualisation, networking,
Bitcoin}\label{lab---virtualisation-networking-bitcoin}}

\hypertarget{overview}{%
\subsection{Overview}\label{overview}}

This document details the process of creating the system detailed in the
accompanying book. It is intended to be complete. It is a how to guide.

\hypertarget{summary-of-software}{%
\subsubsection{Summary of software}\label{summary-of-software}}

Summarise the software and functionality

\hypertarget{prerequisites}{%
\subsection{Prerequisites}\label{prerequisites}}

Ensure that the BIOS / firmware / etc of the hardware you intend to use
is up to date.

\hypertarget{network-details}{%
\subsection{Network details}\label{network-details}}

In the example setup provided here there are currently two networks:

\begin{enumerate}
\def\labelenumi{\arabic{enumi}.}
\tightlist
\item
  The virtual server resides in a LAN with the following details:
\end{enumerate}

192.168.x.0/24

Replace x with an integer between 0 and 254

This LAN has a gateway to the Internet and DNS server configured. Of
course, it could be replaced with a direct connection to the Internet,
though for research and development purposes it is often better to work
within a clean LAN and manage access to the Internet as required.

\begin{enumerate}
\def\labelenumi{\arabic{enumi}.}
\setcounter{enumi}{1}
\tightlist
\item
  There is a virtual network configured on the virtual machine host upon
  which virtual machines can reside:
\end{enumerate}

This virtual network is not configured to bridge with the physical
network adapter rather a virtual machine is configured as a gateway to
route IP traffic through. This provides a level of isolation. More on
this later (@todo).

\hypertarget{server-configuration}{%
\subsection{Server configuration}\label{server-configuration}}

\hypertarget{server-hardware-details}{%
\subsubsection{Server hardware details}\label{server-hardware-details}}

@todo

\hypertarget{disk-configuration-details}{%
\subsubsection{Disk configuration
details}\label{disk-configuration-details}}

@todo

\hypertarget{proxmox-ve}{%
\subsection{Proxmox VE}\label{proxmox-ve}}

\hypertarget{installation-and-configuration}{%
\subsubsection{Installation and
configuration}\label{installation-and-configuration}}

Version used: 7.1

Keep in mind that this setup uses the Proxmox VE installer
(https://www.proxmox.com/en/proxmox-ve/get-started) which, as noted on
the site, is a bare-metal installer and will erase all data on at least
one disk. There are alternative methods to install Proxmox VE but these
are not covered here.

A brief summary of the steps taken using Proxmox VE 7.1:

\hypertarget{dialogue-1}{%
\paragraph{Dialogue 1}\label{dialogue-1}}

Choose the target harddisk (/dev/sda in this case).

\hypertarget{dialogue-2}{%
\paragraph{Dialogue 2}\label{dialogue-2}}

Select country, time zone, keyboard layout.

\hypertarget{dialogue-3}{%
\paragraph{Dialogue 3}\label{dialogue-3}}

Set a password (this is the root password, see proxmox hardening
section), and email address.

\hypertarget{dialogue-4}{%
\paragraph{Dialogue 4}\label{dialogue-4}}

Select a Network Interface Card (NIC) on which the management interface
will be available and provide a hostname, IP address, gateway and DNS
server.

In this example the following settings were used:

Hostname: proxmoximus.local IP address: 192.168.x.220 / 24 Gateway:
192.168.x.254 DNS server: 192.168.x.254

Either replace x with the integer used earlier and update the last octet
of the gateway and server with that that corresponds to your setup
(assuming the setup is local and has a local dns server or forwarder) or
configure the values according to your intended setup.

Once the install has completed and the system has rebooted it is time to
begin configuring. This is done (almost entirely) via the web interface,
in this case, available at https:// proxmoximus \textbar{} 192.168.x.220
: 8006

It is also possible to login to a shell via the local terminal and SSH
(which is enabled by default @todo: in hardening, add keys and remove
ability to login with password).

\hypertarget{software-updates}{%
\subsubsection{Software updates}\label{software-updates}}

If you are in a testing and non-production environment then it is
possible access updates without a subscription as detailed here:
https://pve.proxmox.com/wiki/Package\_Repositories. Update
\texttt{/etc/apt/sources.list} as detailed under the Proxmox VE
No-Subscription Repository. This can be achieved via the local terminal,
SSH or web interface (via Shell option).

For example, edit the file:

\texttt{nano\ /etc/apt/sources.list}

Add the following:

\begin{verbatim}
# PVE pve-no-subscription repository provided by proxmox.com,
# NOT recommended for production use
deb http://download.proxmox.com/debian/pve bullseye pve-no-subscription
\end{verbatim}

To the existing:

\begin{verbatim}
deb http://ftp.uk.debian.org/debian bullseye main contrib

deb http://ftp.uk.debian.org/debian bullseye-updates main contrib

# security updates
deb http://security.debian.org bullseye-security main contrib
\end{verbatim}

Resulting in:

\begin{verbatim}
deb http://ftp.debian.org/debian bullseye main contrib
deb http://ftp.debian.org/debian bullseye-updates main contrib

# PVE pve-no-subscription repository provided by proxmox.com,
# NOT recommended for production use
deb http://download.proxmox.com/debian/pve bullseye pve-no-subscription

# security updates
deb http://security.debian.org/debian-security bullseye-security main contrib
\end{verbatim}

The Proxmox VE system will now retrieve updates for both itself and the
base Debian system.

Then from a shell run:

\begin{verbatim}
$ apt update
$ apt upgrade
\end{verbatim}

@todo: determine if system needs a reboot

\hypertarget{proxmox-ve-hardening}{%
\subsubsection{Proxmox VE hardening}\label{proxmox-ve-hardening}}

Links

@todo

Adding users

Add SSH keys and remove ability to login with password

\hypertarget{setup-an-internal-only-network-in-proxmox-ve}{%
\subsection{Setup an internal only network in Proxmox
VE}\label{setup-an-internal-only-network-in-proxmox-ve}}

From the Web GUI navigate to Datacenter - \textgreater{} your server
-\textgreater{} Network

From the menu select Create then Linux Bridge

Input the desired IPv4/CIDR in this case 192.168.y.0/24 and add a
comment if desired (``Internal network'' was used here). Note that y
must not be the same as x previously used.

Name was left as vmbr1

Credit:
https://dannyda.com/2020/06/01/how-to-create-an-internal-only-isolated-network-for-guest-os-virtual-machines-vm-on-proxmox-ve-pve-like-in-vmware-workstation-host-only-network-but-different/

\hypertarget{install-and-configure-internet-gateway-server-virtual-machine}{%
\subsection{Install and configure Internet gateway server virtual
machine}\label{install-and-configure-internet-gateway-server-virtual-machine}}

VyOS was selected (https://vyos.io/)

\hypertarget{create-an-iso-of-the-stable-version-as-of-writing-1.3.0}{%
\subsubsection{Create an ISO of the stable version (as of writing
1.3.0)}\label{create-an-iso-of-the-stable-version-as-of-writing-1.3.0}}

@todo: the built version seemed to be a nightly release, is it possible
to add a tag to get a stable build?

Follow the build instructions:

https://docs.vyos.io/en/latest/contributing/build-vyos.html

This document does not list this version (goes up to 10 ``buster'') but
Debian 11 ``bullseye'' was successfully used in this setup.

Run the following commands:

\begin{verbatim}
$ apt install git
$ apt install build-essential
\end{verbatim}

Follow the instructions here
https://docs.docker.com/engine/install/debian/ to install Docker

Run the following commands:

\begin{verbatim}
$ git clone -b equuleus --single-branch https://github.com/vyos/vyos-build

$ docker run --rm -it --privileged -v $(pwd):/vyos -w /vyos vyos/vyos-build:equuleus bash
\end{verbatim}

Then in the Docker terminal run the following commands:

\begin{verbatim}
./configure --architecture amd64

sudo make iso
\end{verbatim}

\hypertarget{upload-the-iso-image-to-the-proxmox-ve-server}{%
\subsubsection{Upload the ISO image to the Proxmox VE
server}\label{upload-the-iso-image-to-the-proxmox-ve-server}}

\begin{enumerate}
\def\labelenumi{\arabic{enumi}.}
\item
  Via the web GUI navigate to Datacenter -\textgreater{} your server
  -\textgreater{} local.
\item
  In the right hand pane select ISO Images and then upload.
\item
  Upload the ISO image
\end{enumerate}

Tip: you can also pass the checksum to the Proxmox VE upload tool

\hypertarget{create-vyos-virtual-machine}{%
\subsubsection{Create VyOS virtual
machine}\label{create-vyos-virtual-machine}}

\begin{enumerate}
\def\labelenumi{\arabic{enumi}.}
\item
  From the top right of the web GUI select Create VM
\item
  In the appearing dialogue type a Name ``VyOS'' and optionally select
  advanced and Start at boot
\item
  On the next tab select the target ISO image
\item
  On the System tab leave everything as default
\item
  n the Disk tab leave the defaults (this exceeds requirements
  https://docs.vyos.io/en/latest/installation/install.html)
\item
  On the CPU tab:

  Sockets: 1, Cores: 2
\item
  On the Memory tab

  Memory: 4096MiB
\item
  On the Network tab

  Choose the bridge with the internet vmbr0 (it is possible to add the
  second later) and leave the defaults including firewall

  Confirm all the settings on the next tab but \textbf{do not} select
  start after created

  Navigate to the newly created VM on the left-hand pane then selected
  Hardware from the menu that is presented on the right. Choose Add and
  then Network Device. In the dialogue that appears select the Internal
  network bridge (vmbr1 in this case) that was created earlier and leave
  all other options as is.

  So, the VM will have the following Network Devices:

  net0: Internet

  net1: Internal only
\item
  Start the VM and connect the console (top right)
\item
  Login with vyos and vyos

  Run the command:

\begin{verbatim}
$ install image
\end{verbatim}
\item
  Follow the instructions
\item
  Set the CD/DVD to none in Web GUI
\item
  Reboot
\end{enumerate}

\hypertarget{configure-vyos}{%
\subsubsection{Configure VyOS}\label{configure-vyos}}

Open a noVNC window to the host

Login with vyos and vyos

Switch to configure mode:

\begin{verbatim}
vyos@vyos$ configure
vyos@vyos#
\end{verbatim}

Then configure as desired. Below is configuration used in the setup here
(if you use for inspiration do take care to replace the x and y octet
values correctly with previously chosen values. The z octet value should
be something unused in the outside LAN for which the host is physically
connected):

\begin{verbatim}
set interfaces ethernet eth0 address '192.168.x.z/24'
set interfaces ethernet eth0 description 'OUTSIDE'
set protocols static route 0.0.0.0/0 next-hop 192.168.x.254 distance 1
set service dns forwarding system
set service dns forwarding name-server 192.168.x.254
set service dns forwarding listen-address 192.168.y.1
set service dns forwarding allow-from 192.168.y.0/24
set system name -server 192.168.x.254

set interfaces ethernet eth1 address '192.168.y.1/24'
set interfaces ethernet eth1 description 'INSIDE'

set nat source rule 100 outbound-interface eth0
set nat source rule 100 source address 192.168.y.0/24
set nat source rule 100 translation address masquerade

set service ssh listen-address 0.0.0.0
\end{verbatim}

Once done remember to commit the config (correcting any
misconfiguration) and save.

\begin{verbatim}
commit
save
\end{verbatim}

Inspiration for the above was taken from:
https://bertvv.github.io/cheat-sheets/VyOS.html

@todo: hardening, IDS, IPS

\hypertarget{install-and-configure-a-debian-virtual-machine}{%
\subsection{Install and configure a Debian virtual
machine}\label{install-and-configure-a-debian-virtual-machine}}

This VM can be used for various tasks such as software compilation and
testing of the networks. In this setup the Debian VM was used to test
connectivity to the VyOS gateway and the Internet. It is also used in
the subsequent stages to deploy a nix-bitcoin node.

In Proxmox VE create a new virtual machine and configure the network
device to use the bridge `vmbr1'.

Then install Debian and configure the network adapter within the VM with
the following settings:

IP address: 192.168.y.2 Gateway: 192.168.y.1 DNS: 192.168.y.1

Test that the VM has Internet connectivity.

\hypertarget{deploying-the-nix-bitcoin-node}{%
\subsection{Deploying the nix-bitcoin
node}\label{deploying-the-nix-bitcoin-node}}

This deployment follows the documentation:

https://github.com/fort-nix/nix-bitcoin/\#get-started

Take note of the hardware requirements:

https://github.com/fort-nix/nix-bitcoin/blob/master/docs/hardware.md

In the main, the install guide
(https://github.com/fort-nix/nix-bitcoin/blob/master/docs/install.md) is
followed verbatim and notes with a reference to particular sections are
added where appropriate.

Optional - a small exception in regards to this setup is that a separate
virtual disk (located on a different physical drive mirror (RAID 1)) was
used to store the bitcoin database - this is optional and details are
provided on how to achieve it. Also detailed is how to configure the
network when using the minimal image.

\hypertarget{acquiring-nixos}{%
\subsubsection{Acquiring NixOS}\label{acquiring-nixos}}

Following
\href{https://github.com/fort-nix/nix-bitcoin/blob/master/docs/install.md\#1-nixos-installation}{section
1.1} make sure the latest NixOS is obtained i.e.~do not just copy the
whole wget command outright and make sure to verify the hash against
trusted sources before using the image.

Download the minimal ISO image (https://nixos.org/download.html)

Verify the hash

Upload the ISO to
\protect\hyperlink{ux5cux23ux5cux23Upload-the-ISO-image-to-the-Proxmox-VE-server}{Proxmox
VE server}

\hypertarget{create-a-new-vm}{%
\subsubsection{Create a new VM}\label{create-a-new-vm}}

Name: NixOS

Follow the setup and leave everything as default until the CPU page. The
following configuration was used, which should exceed the minimum
requirements:

Cores: 4

Memory: 4096MiB = 4.2GB

Network: vmbr1 (Internal Network)

Do NOT check the select the start the VM checkbox

Next, an additional drive will be configured in Proxmox VE. This will
then be used to store the bitcoin database within the NixOS VM.

Select Datacenter -\textgreater{} server name and then from the right
pane Disks -\textgreater{} LVM-Thin. Then select Create: Thinpool

From the dialogue select the disk and type a name ``data'' was used in
this setup. This provisions a vg with the name \emph{data} and a name
\emph{data} @todo: review

Navigate back to the VM created and choose Hardware and then Add
-\textgreater{} Hard Disk

Choose ``data'' from Storage and then set the size to 560 GiB which
equates to about 600GB

Now, continue from section 1.3 in the install instructions

Start the VM and connect a console

\texttt{sudo\ -i}

With the SeaBios that was used in this setup the file does not exist and
Legacy Boot (MBR) should be followed (option 2)

Note: no consideration is currently given for encrypted partitions
within the Proxmox VE setup

Enable the OpenSSH daemon

\begin{verbatim}
services.openssh.permitRootLogin = "yes";
\end{verbatim}

Configure the network config in configuration.nix (remember to replace y
with the chosen value)

\begin{verbatim}
  networking.useDHCP = false;
  networking.interfaces.ens18.useDHCP = false;

  networking.interfaces.ens18.ipv4.addresses = [ {
    address= "192.168.y.3";
    prefixLength = 24;
  } ];
  networking.defaultGateway = "192.168.y.1";
  networking.nameservers = ["192.168.y.1"];
  networking.hostName = "nixicon";
\end{verbatim}

Although the IP above will be assigned once the nix-bitcoin is deployed
the installation cannot continue without a connection to the Internet so
that needs to be configured:

\begin{verbatim}
$ ifconfig ens18 192.168.y.3
$ ifconfig ens18 255.255.255.0
$ ip route add 192.168.y.0/24 dev ens18 scope link src 192.168.y.3
\end{verbatim}

Then add the nameserver:

\begin{verbatim}
nano /etc/resolv.conf
\end{verbatim}

Add:

\begin{verbatim}
nameserver 192.168.y.1
\end{verbatim}

Once the above is complete and successful networking is verified

Run the following command:

\texttt{nixos-install}

Set the root password and then reboot.

\hypertarget{configure-the-additional-drive-optional}{%
\subsubsection{Configure the additional drive
(optional)}\label{configure-the-additional-drive-optional}}

As the additional drive was not configured at the time of the install
then the parted utility will need to be available. To achieve this, edit
the configuration.nix file

\texttt{nano\ /etc/nixos/configuration.nix}

and add the following:

\begin{verbatim}
environment.systemPackages = with pkgs; [
    parted
];
\end{verbatim}

Then issue the following command:

\texttt{nixos-rebuild\ switch}

Determine the desired drive, fdisk can assist:

\texttt{fdisk\ -l}

Note: in this sytem the desired drive is /dev/sdb with 560GiB capacity
but sdx is used in the following examples:

Then partition:

\texttt{parted\ /dev/sdx}

\begin{verbatim}
(parted) mklabel msdos
(parted) mkpart primary
File system type? ext4
Start? 0%
End? 100%
quit
\end{verbatim}

(note: it is possible to combine the above as a single line command)

Then create the file system:

\texttt{mkfs.ext4\ -L\ data\ /dev/sdx1}

Make a note of the UUID as this will be used in the next steps to mount
the volume

\hypertarget{create-port-forwarding-rules-for-ssh-optional}{%
\subsubsection{Create port forwarding rules for SSH
(optional)}\label{create-port-forwarding-rules-for-ssh-optional}}

Providing SSH access to the VMs from outside the private network makes
it easier to configure them (ability to copy and paste UUIDs etc.)

This involve updates to VyOS configuration and can be temporary.

Login to the vyos, you should be able do this from your local machine
now as apposed to the console

ssh vyos@192.168.x.z

\hypertarget{debian}{%
\paragraph{Debian}\label{debian}}

192.168. y .2

The following commands were issued to the VyOS router (obiously
replacing y with the value chosen earlier)

\begin{verbatim}
configure

set nat destination rule 12 description 'Port Forward: 2222 to 22 SSH on 192.168.y.2'
set nat destination rule 12 destination port '2222'
set nat destination rule 12 inbound-interface 'eth0'
set nat destination rule 12 protocol 'tcp'
set nat destination rule 12 translation address '192.168.y.2'
set nat destination rule 12 translation port '22'

commit
\end{verbatim}

Now test

Note: for the Debian VM the user account may need to be added to the SSH
user group

Note: you could SSH from Debian to all other hosts

\hypertarget{nixos}{%
\paragraph{NixOS}\label{nixos}}

192.168. y .3

Assuming access to the Debian VM via SSH is working then from the same
VyOs configure session issue the following:

\begin{verbatim}
set nat destination rule 13 description 'Port Forward: 2223 to 22 SSH on 192.168.y.3'
set nat destination rule 13 destination port '2223'
set nat destination rule 13 inbound-interface 'eth0'
set nat destination rule 13 protocol 'tcp'
set nat destination rule 13 translation address '192.168.y.3'
set nat destination rule 13 translation port '22'

commit
\end{verbatim}

Test and if all is well, save the VyOS configuration:

\begin{verbatim}
save
\end{verbatim}

Credit: https://support.vyos.io/en/kb/articles/nat-principles

Having SSH access to both the Debian and NixOS VMs will make the next
stages of the process a little easier

@todo hardening (SSH e.g.~add keys, remove plain text or remove SSH
access entirely)

\hypertarget{prepare-nix-bitcoin-nixos-package}{%
\subsubsection{Prepare nix-bitcoin NixOS
package}\label{prepare-nix-bitcoin-nixos-package}}

This section continues to follow the guide from
\href{https://github.com/fort-nix/nix-bitcoin/blob/master/docs/install.md}{Nix
Installation}.

Note: this part of the guide will be executed on the Debian VM that was
installed earlier

The next steps will follow
\href{https://github.com/fort-nix/nix-bitcoin/blob/master/docs/install.md\#2-nix-installation}{section
2}.

You may need to add your user to the sudoers if it is not a member
already

In Debian this can be achieved with the following commands

Switch to root

\texttt{su}

Then

\texttt{sudo\ usermod\ -a\ -G\ sudo\ username}

Exit both the root and user session and then log back in as the user

Important: ensure that when downloading the multi-user NixOS that the
latest is obtained (listed at https://nixos.org/download.html). I.e.
dont just copy and paste verbatim.

Note: It is possible to determine the latest version by navigating to:
https://nixos.org/nix/install and this will redirect to for example:
https://releases.nixos.org/nix/nix-2.6.0/install at the time of writing.
From here you could quickly santiy check the redirect by heading to:
https://releases.nixos.org/?prefix=nix/

You could (as in the example on the NixOS website) use curl with a -L
option which will ignore the redirect

Enter a directory to receive the files. \textasciitilde/Downloads was
chosen for this setup

For completeness the following commands were issued:

\texttt{curl\ -o\ install-nix-2.6.0\ https://releases.nixos.org/nix/nix-2.6.0/install}

with the -o option writing the contents to a file rather than displaying
on screen

then

\texttt{curl\ -o\ install-nix-2.6.0.asc\ https://releases.nixos.org/nix/nix-2.6.0/install.asc}

then

\texttt{gpg2\ -\/-keyserver\ hkps://keyserver.ubuntu.com\ -\/-recv-keys\ B541D55301270E0BCF15CA5D8170B4726D7198DE}
\texttt{gpg2\ -\/-verify\ ./install-nix-2.6.0.asc}

Which are similarly detailed here:
https://nixos.org/download.html\#nix-verify-installation

Note: it is not required to run the script as sudo. It will prompt for
permission.

In this setup the:

\begin{verbatim}
substitute = false
\end{verbatim}

was added to /etc/nix/nix.conf as detailed.

Run the script.

Exit the terminal and login in again as per the message:

\begin{verbatim}
Try it! Open a new terminal, and type:

  $ nix-shell -p nix-info --run "nix-info -m"
\end{verbatim}

The next part continues with
\href{https://github.com/fort-nix/nix-bitcoin/blob/master/docs/install.md\#3-setup-deployment-directory}{setting
up the deployment directory}

Stood in the home directory or one just off it, follow the instructions
provided.

Once the above is complete continue with the
\href{https://github.com/fort-nix/nix-bitcoin/blob/master/docs/install.md\#4-deploy-with-krops}{deploy
with krops} section.

Follow the instructions and edit the SSH config. You will need a
public/private key pair for this and this
\href{https://www.digitalocean.com/community/tutorials/how-to-set-up-ssh-keys-2}{article}
could be useful.

The config file used in this setup is shown below:

\begin{verbatim}
Host nixicon
    # FIXME
    Hostname 192.168.y.3
    User root
    PubkeyAuthentication yes
    # FIXME
    IdentityFile ~/.ssh/id_ed25519
    AddKeysToAgent yes
\end{verbatim}

And for reference the krops/deploy.nix is as follows:

\begin{verbatim}
let
  # FIXME:
  target = "root@nixicon";

  extraSources = {
    "hardware-configuration.nix".file = toString ../hardware-configuration.nix;
  };

  krops = (import <nix-bitcoin> {}).krops;
in
krops.pkgs.krops.writeDeploy "deploy" {
  inherit target;

  source = import ./sources.nix { inherit extraSources krops; };

  # Avoid having to create a sentinel file.
  # Otherwise /var/src/.populate must be created on the target node to signal krops
  # that it is allowed to deploy.
  force = true;
}
\end{verbatim}

In subsection 3 the guide shows how to optionally disallow substitutes.
This was set to true in this setup.

In subsection 4 the guide details copying hardware-configuration.nix
file to the deployment directory and then in subsection 5 making edits
to the configuration.nix file to turn on desired modules. There are some
important notes relevant to this setup to make here:

\hypertarget{additional-hard-drive-configuration}{%
\paragraph{Additional hard drive
configuration}\label{additional-hard-drive-configuration}}

No edits were made to hardware-configuration.nix as per the warning at
the top of the file. For reference here is the file from this setup:

\begin{verbatim}
# Do not modify this file!  It was generated by ‘nixos-generate-config’
# and may be overwritten by future invocations.  Please make changes
# to /etc/nixos/configuration.nix instead.
{ config, lib, pkgs, modulesPath, ... }:

{
  imports =
    [ (modulesPath + "/profiles/qemu-guest.nix")
    ];

  boot.initrd.availableKernelModules = [ "ata_piix" "uhci_hcd" "virtio_pci" "virtio_scsi" "sd_mod" "sr_mod" ];
  boot.initrd.kernelModules = [ ];
  boot.kernelModules = [ ];
  boot.extraModulePackages = [ ];
  boot.loader.grub.device = "/dev/sda";

  fileSystems."/" =
    { device = "/dev/disk/by-uuid/UUID_1";
      fsType = "ext4";
    };

  swapDevices =
    [ { device = "/dev/disk/by-uuid/UUID_2"; }
    ];

  hardware.cpu.intel.updateMicrocode = lib.mkDefault config.hardware.enableRedistributableFirmware;
}
\end{verbatim}

Rather, the additional hard drive was configured in the
configuration.nix as shown here:

\begin{verbatim}
fileSystems."/var/lib" =
{ device = "/dev/disk/by-uuid/UUID_3";
  fsType = "ext4";
};
\end{verbatim}

This mounts /var/lib (which contains the bitcoin database etc.) to the
additional drive.

\hypertarget{static-ip-configuration}{%
\paragraph{Static IP configuration}\label{static-ip-configuration}}

To configure the static IP add the following:

\begin{verbatim}
networking.useDHCP = false;
networking.interfaces.ens18.useDHCP = false;

networking.interfaces.ens18.ipv4.addresses = [ {
  address= "192.168.y.3";
  prefixLength = 24;
} ];
networking.defaultGateway = "192.168.y.1";
networking.nameservers = ["192.168.y.1"];
networking.hostName = "nixicon";
\end{verbatim}

\hypertarget{ssh-configuration}{%
\paragraph{SSH configuration}\label{ssh-configuration}}

Below is the snipet of configuration. Note: paste the contents of
\texttt{\textasciitilde{}/.ssh/id\_ed25519.pub} where the
\texttt{\#\ FIXME:\ Replace\ this\ with\ your\ SSH\ pubkey\ appears}

\begin{verbatim}
services.openssh = {
  enable = true;
  passwordAuthentication = false;
};
users.users.root = {
  openssh.authorizedKeys.keys = [
    # FIXME: Replace this with your SSH pubkey
    "ssh-ed25519 LONG_KEY user@debian"
  ];
};
\end{verbatim}

\hypertarget{services-configuration}{%
\paragraph{Services configuration}\label{services-configuration}}

Last but not least, the following services are enabled in this setup:

\begin{verbatim}
services.clightning.enable = true;
services.rtl.enable = true;
services.rtl.nodes.clightning = true;
services.electrs.enable = true;
services.backups.enable = true;
\end{verbatim}

Once the configuration.nix file has been updated continue from
subsection 6.
